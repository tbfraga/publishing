% MODELO CONEM 2022 - Obs. Adequado para o CONEM 2024.
%
% Date: January 12, 2024
%
\documentclass[10pt,fleqn,a4paper,twoside]{article}
\usepackage{abcm}
\def\shortauthor{T. B. Fraga, B. M. Cavalcanti, A. M. D. Silva e E. L. R. Silva \textcolor[rgb]{0.98,0.00,0.00}}
\def\shorttitle{ABC Multicriteria Classification with Analytical Hierarchy Process: How to Force Consistency of Paiwise Comparisons Matrix \textcolor[rgb]{0.98,0.00,0.00}}

\usepackage{algpseudocode}

\usepackage{array, makecell}

\usepackage[table]{xcolor}
\usepackage{tikz}
\usetikzlibrary{shapes.geometric,shapes.misc,shapes.symbols,arrows.meta,graphs,fit,positioning,shadows}

\begin{document}
    
    % CABEÇALHO
    \thispagestyle{empty}
    \begin{figure}[h]
        \begin{center}
            \includegraphics[angle=0, width=\textwidth]{Logo_template.png}
        \end{center}
    \end{figure}
    \vspace{-.5cm}
    \hspace{-.8cm}
    \begin{tabular}{||p{\textwidth}}
    \begin{center}
    \vspace{-.6cm}
    %
    % Digitar o título
    \title{CONEM2024-0025\\ ABC  MULTICRITERIA CLASSIFICATION WITH ANALYTICAL HIERARCHY PROCESS: HOW TO FORCE CONSISTENCY OF PAIRWISE COMPARISONS MATRIX} 
    %
    \end{center}
	%
	% Digitar os nomes dos autores/instituição
    \authors{Tatiana Balbi Fraga, tatiana.balbi@ufpe.br$^1$} \\
    \authors{Beatriz Marinho Cavacanti, beatriz.marinhocavalcanti@ufpe.br$^1$} \\
    \authors{Alexia Maria Duque Silva, alexia.duque@ufpe.br$^1$} \\
    \authors{Erika Leticia Rodrigues Silva, erika.leticias@ufpe.br$^1$} \\\\
    \institution{$^1$Nome da instituição, endereço para correspondência (institucional, se houver)} \\
    \\
    %
    % Digitar o resumo
    \abstract{\textbf{Resumo:} Multicriteria classification is usually a crucial step in the decision-making in the manufacturing management process. For such classification, the attribution of weights to the criteria strongly influences the coherence of the categorization. Saaty's Analytic Hierarchy Process (AHP) is an eminent method for assigning weights to multiple criteria. AHP's logic is not complicated at all. However, since criteria's pairwise comparison matrices are usually generated manually and based only on some employee know-how, there is high complexity in elaborating a consistent pairwise matrix, especially when comparing several criteria. This paper presents a constructive algorithm that can adjust inconsistent matrices, forcing such matrices to have a better consistency rate. We tested this algorithm by applying the AHP method for ABC multicriteria classification to companies in three sectors. As a result, we observed that the algorithm adjusts the pairwise matrices in just a few seconds, avoiding the manual work of several hours, then showing that it is an essential resource for applying the AHP method.}\\
    %
    % Digitar as palavras-chave
   \\
    \keywords{\textbf{Palavras-chave:} ABC multicriteria classification, analytic hierarchy process, consistency rate,  constructive algorithm, COPSolver library}\\
    %
    \end{tabular}
    
    \section{INTRODUCTION}
    
    \subsection{Pairwise Comparisons Matrix and Consistency Rate}
    
    \section{ALGORITHM FOR FORCING LOW CONSISTENCY RATE}
    
    \section{ON MULTICRITERIA ABC CLASSIFICATION}

In our study, we observed that one of the crucial points for correctly applying multicriteria ABC classification is correctly attributing the weights for the initial ABC classification.

When we built the input data file, we considered three types of criteria for ABC analysis, cumulative sum (for billing and lead time criteria), qualitative, and binary criteria. While we were testing the solver, we noticed that an incorrect assignment of the weights can generate a strong bias in the multicriteria classification, since the criterion with the top percentage difference will always be dominant.

Therefore, to assign the weights, three procedures were adopted, according to the different nature of each type of criterion used.

In the case of the binary criteria, the choice of a binary parameters (0 and 1) results in a distribution of the percentage among the criteria that are critical. This seems to be consistent with the desired goal. However, in the case of the other criteria types, the consistency is no longer so clear. 

We can verify this inconsistency through a brief analysis of the two criteria classified according to the cumulative sum. For the data collected from the furniture trades company, the product with the highest billing (242,280.00 reais) has a billing corresponding to 12\% of the total amount; and a product with a longer lead time (30 days) represents 0.46\% of the total lead time, that is, a percentage 0.25\% higher than products with a lead time of 20 days (0.31\%) and 0.35\% higher than products with a lead time of 7 days (0.11\%).  

The question is: is the difference between the importance of a product with a 30-day lead time and a product with a 7-day lead time only 0.35\%? This value can only be appreciated when compared with the difference in importance of 12\% between the product with the highest and the product with the lowest billing. 

Make a multicriteria classification for the billing and lead time criteria based only on the accumulated sum is like calculate a weighted average of the annual sales of vehicles and bananas, hoping that this average will have some meaning. For there to be real meaning in the weighted average found by the multicriteria method, it is necessary to apply some procedure to balance the weights assigned to each criterion.
	
	\citet{Odu2019} cites CRITIC as one of the methods applied for weight assignment in multicriteria classification. This method starts with normalizing the weights using the following equations:
	
	\begin{equation}
		\rho_{ij} = \frac{w_{ij} - w_{j}^{\min}}{w_{j}^{\max} - w_{j}^{\min}} \qquad i=1,...,m; j = 1, ..., n \qquad \textrm{for benefit criteria}
	\end{equation}
	
	\begin{equation}
		\rho_{ij} = \frac{w_{j}^{\max} - w_{ij}}{w_{j}^{\max} - w_{j}^{\min}} \qquad i=1,...,m; j = 1, ..., n \qquad \textrm{for cost criteria}
	\end{equation}
	
	Where $w_{j}^{\min}$ and $w_{j}^{\max}$ are, respectively, the lowest and highest weights assigned to criterion $j$, $w_{ij}$ is the weight and $\rho_{ij}$ is the normalized weight of criteria $j$ for product $i$. 
	 
	The problem with this normalization method is that the largest values of the different criteria will have exactly the same weight. In the case of the companies studied, this may not correspond to the assessment of experts. 
	
	The solution that we adopted in this work for multicriteria classification with lead time and billing criteria, was based on the work of Willians. A single measure was adopted for these two criteria (lead time billing) dividing the billing by the lead time.

In the case of qualitative criteria, the initial ABC classification was made using three weights, one weight assigned to products A, pa, a second weight assigned to products B, pb, and a third weight assigned to products C, pc, where pa > pb > pc. For qualitative criteria, pa, pb and pc must be assigned using the following steps:
	
\begin{itemize}
\item order the products, in descending order, according to their respective values for lead time billing;
\item classify products into three groups, A, B and C, for each qualitative criterion;
\item do pc = 0 for all group C products, for each qualitative criterion;
\item for each qualitative criterion, do peer-to-peer comparison with the groups A and B and the products ordered according to the lead time billing criterion, answering the following question: the degree of importance of the products of this group to the qualitative criterion is equivalent to what degree of importance of the other product evaluated according to the lead time billing criterion ?;
\item assign to pa and pb the value of the lead time billing of the product with corresponding importance.
\end{itemize}	

Equation (\ref{eq:weight}) can be used for calculation of multicriteria classification weights of each product, $\rho_{p}$:

\begin{equation}
	\label{eq:weight}
	\rho_{p} = (wb + wlt) * \frac{b_p}{lt_p} + \sum_{q=1}^{\textrm{NQ}}{w_{\sigma_q}*\frac{b_{e_{p\sigma_q}}}{lt_{e_{p\sigma_q}}}} + \sum_{b=1}^{\textrm{NB}}{w_{\beta_b}*w_{p\beta_b}}, \quad p= 1,...,\textrm{NP}
\end{equation}

where:

$\textrm{NP}$ is the number of products;

$\textrm{NQ}$ is the number of qualitative criteria;

$\textrm{NB}$ is the number of binary criteria;

$\Sigma$ is the set of qualitative criteria, \quad $\Sigma = \{\sigma_q\}_{q=1}^{NQ}$;

$\textrm{B}$ is the set of binary criteria, \quad $\textrm{B} = \{\beta_b\}_{b=1}^{NB}$;

$p$ is the index used to represent each product, \quad $p= 1,...,\textrm{NP}$;

$e_{p\sigma_q}$ is the product lead time billing equivalent to product $p$ on the criterion $\sigma_q$;

$c$ is the index used to represent each criterion of qualitative or binary type , \quad $c \in \Sigma \cup \textrm{B}$;
 
$b_p$ is the billing of product $p$;

$lt_p$ is the lead time of product $p$;

$\frac{b_p}{lt_p}$ is the lead time billing of product $p$;

$w_{pc}$ is the weight of the product $p$ according to criterion $c$;

$wb$ is the weight of criterion billing.

$wlt$ is the weight of criterion lead time.

$w_{c}$ is the weight of criterion $c$.

    
    \section{RESULTS AND DISCUSSIONS}
    
    \subsection{Analysis on the Algorithm for Adjusting Inconsistencies}
    
    Table \ref{tab:pairwiseMatrix} presents the modifications made by the software 'COPSolver: library for solving the multicriteria classification problem' in the pairwise comparisons matrix and the consequent changes in the consistency rate (CR) for the data collected from the three companies selected for this study.

	\begin{table}[ht]
            \begin{center}
                \caption{\textbf{Pairwise comparisons matrix and consistency rate changes for three companies 					(results found by COPSolver) - legend: sb = substitutability; lt = lead-time; rp = repairability; cr = criticality; ob = obsolescence; bl = billing; cm = commonality; CR = consistence rate; w = normalized weights vector.}}
                    \begin{tabular}[l]{p{0.4cm} p{0.5cm} p{0.5cm} p{0.5cm} p{0.5cm} p{0.5cm} p{0.5cm} p{0.5cm} | p{0.5cm} p{0.5cm} p{0.5cm} p{0.5cm} p{0.5cm} p{0.5cm} p{0.5cm}}
						\multicolumn{14}{l}{car mechanics company} \\
						\cline {1-15} \\
						& \multicolumn{7}{c | }{CR = 0.216}  & \multicolumn{7}{c}{CR = 0.085} \\
						& \multicolumn{7}{c | }{$w=(0.35, 0.26, 0.18, 0.11, 0.05, 0.03, 0.01)$}  & \multicolumn{7}{c}{$w = (0.37, 0.26, 0.12, 0.12, 0.06, 0.05, 0.02)$} \\
   						& sb & lt & rp & cr & ob & bl & cm & sb & lt & rp & cr & ob & co & cm \\
						sb & 1.00 & 3.00 & 3.00 & \cellcolor[HTML]{ACE600} 5.00 & 7.00 & 9.00 & 9.00 & 1.00 & 3.00 								& 3.00 & \cellcolor[HTML]{ACE600} 3.00 & 7.00 & 9.00 & 9.00 \\
						lt & 0.33 & 1.00 & 3.00 & \cellcolor[HTML]{ACE600} 5.00 & \cellcolor[HTML]{ACE600} 7.00 & 9.00 & 9.00 & 0.33 & 1.00 & 3.00 & \cellcolor[HTML]{ACE600} 3.00 & \cellcolor[HTML]{ACE600} 5.00 & 9.00 & 9.00 \\
						rp & 0.33 & 0.33 & 1.00 & \cellcolor[HTML]{ACE600} 3.00 & \cellcolor[HTML]{ACE600} 9.00 & \cellcolor[HTML]{ACE600} 9.00 & 9.00 & 0.33 & 0.33 & 1.00 & \cellcolor[HTML]{ACE600} 1.00 & \cellcolor[HTML]{ACE600} 3.00 & \cellcolor[HTML]{ACE600} 3.00 & 9.00 \\
						cr & \cellcolor[HTML]{ACE600} 0.20 & \cellcolor[HTML]{ACE600} 0.20 & \cellcolor[HTML]{ACE600} 0.33 & 1.00 & \cellcolor[HTML]{ACE600} 5.00 & \cellcolor[HTML]{ACE600} 9.00 & 9.00 & \cellcolor[HTML]{ACE600} 0.33 & \cellcolor[HTML]{ACE600} 0.33 & \cellcolor[HTML]{ACE600} 1.00 & 1.00 & \cellcolor[HTML]{ACE600} 3.00 & \cellcolor[HTML]{ACE600} 3.00 & 9.00 \\
						ob & 0.14 & \cellcolor[HTML]{ACE600} 0.14 & \cellcolor[HTML]{ACE600} 0.11 & \cellcolor[HTML]{ACE600} 0.20 & 1.00 & \cellcolor[HTML]{ACE600} 5.00 & 9.00 & 0.14 & \cellcolor[HTML]{ACE600} 0.20 & \cellcolor[HTML]{ACE600} 0.33 & \cellcolor[HTML]{ACE600} 0.33 & 1.00 & \cellcolor[HTML]{ACE600} 1.00 & 9.00 \\
						np & 0.11 & 0.11 & \cellcolor[HTML]{ACE600} 0.11 & \cellcolor[HTML]{ACE600} 0.11 & \cellcolor[HTML]{ACE600} 0.20 & 1.00 & 9.00 & 0.11 & 0.11 & \cellcolor[HTML]{ACE600} 0.33 & \cellcolor[HTML]{ACE600} 0.33 & \cellcolor[HTML]{ACE600} 1.00 & 1.00 & 9.00 \\
						cm & 0.11 & 0.11 & 0.11 & 0.11 & 0.11 & 0.11 & 1.00 & 0.11 & 0.11 & 0.11 & 0.11 & 0.11 & 0.11 & 1.00 \\
						\cline {1-15} \\
						\multicolumn{14}{l}{furniture trades company} \\
						\cline {1-15} \\
  						& \multicolumn{7}{c | }{CR = 0.151}  & \multicolumn{7}{c}{CR = 0.096} \\
  						& \multicolumn{7}{c | }{$w=(0.05, 0.10, 0.14, 0.21, 0.02, 0.45, 0.02\underline{})$}  & \multicolumn{7}{c}{$w = (0.07, 0.16, 0.25, 0.20, 0.02, 0.26, 0.03)$} \\
   						& lt & sb & rp & cr & cm & bl & ob & lt & sb & rp & cr & cm & np & ob \\
						lt & 1.00 & 0.20 & 0.14 & \cellcolor[HTML]{ACE600} 0.20 & 7.00 & \cellcolor[HTML]{ACE600} 0.11 & 5.00 & 1.00 & 0.20 & 0.14 & \cellcolor[HTML]{ACE600} 0.33 & 7.00 & \cellcolor[HTML]{ACE600} 0.20 & 5.00 \\
						sb & 5.00 & 1.00 & 0.33 & \cellcolor[HTML]{ACE600} 0.33 & 5.00 &\cellcolor[HTML]{ACE600}  0.14 & 7.00 & 5.00 & 1.00 & 0.33 & \cellcolor[HTML]{ACE600} 1.00 & 5.00 & \cellcolor[HTML]{ACE600} 0.33 & 7.00 \\
						rp & 7.00 & 3.00 & 1.00 & \cellcolor[HTML]{ACE600} 0.33 & 5.00 & \cellcolor[HTML]{ACE600} 0.14 & 3.00 & 7.00 & 3.00 & 1.00 & \cellcolor[HTML]{ACE600} 1.00 & 7.00 & \cellcolor[HTML]{ACE600} 1.00 & 3.00 \\
						cr & \cellcolor[HTML]{ACE600} 5.00 & \cellcolor[HTML]{ACE600} 3.00 & \cellcolor[HTML]		{ACE600} 3.00 & 1.00 & \cellcolor[HTML]{ACE600} 9.00 & \cellcolor[HTML]{ACE600} 0.33 & 9.00 & \cellcolor[HTML]{ACE600} 3.00 & \cellcolor[HTML]{ACE600} 1.00 & \cellcolor[HTML]{ACE600} 1.00 & 1.00 & \cellcolor[HTML]{ACE600} 7.00 & \cellcolor[HTML]{ACE600} 1.00 & 9.00 \\
						cm & 0.14 & 0.20 & \cellcolor[HTML]{ACE600} 0.20 & \cellcolor[HTML]{ACE600} 0.11 & 1.00 & 0.11 & 1.00 & 0.14 & 0.20 & \cellcolor[HTML]{ACE600} 0.14 & \cellcolor[HTML]{ACE600} 0.14 & 1.00 & 0.11 & 1.00 \\
						np & \cellcolor[HTML]{ACE600} 9.00 & \cellcolor[HTML]{ACE600} 7.00 & \cellcolor[HTML]{ACE600} 7.00 & \cellcolor[HTML]{ACE600} 3.00 & 9.00 & 1.00 & 9.00 & \cellcolor[HTML]{ACE600} 5.00 & \cellcolor[HTML]{ACE600} 3.00 & \cellcolor[HTML]{ACE600} 1.00 & \cellcolor[HTML]{ACE600} 1.00 & 9.00 & 1.00 & 9.00 \\
						ob & 0.20 & 0.14 & 0.33 & 0.11 & 1.00 & 0.11 & 1.00 & 0.20 & 0.14 & 0.33 & 0.11 & 1.00 & 0.11 & 1.00 \\
						\cline {1-15} \\
						\multicolumn{14}{l}{plastic packaging manufacturing company} \\
						\cline {1-15} \\
   						& \multicolumn{7}{c | }{CR = 0.625}  & \multicolumn{7}{c}{CR = 0.085} \\
   						& \multicolumn{7}{c | }{$w=(0.39, 0.10, 0.22, 0.09, 0.12, 0.07)$}  & \multicolumn{7}{c}{$w = (0.50, 0.18, 0.14, 0.06, 0.09, 0.03)$} \\
   						& np & cr & lt & ob & sb & rp & & bl & cr & lt & ob & sb & rp & \\
						np & 1.00 & 3.00 & 7.00 & \cellcolor[HTML]{ACE600} 7.00 & 5.00 & \cellcolor[HTML]{ACE600} 5.00 & & 1.00 & 3.00 & 7.00 & \cellcolor[HTML]{ACE600} 9.00 & 5.00 & \cellcolor[HTML]{ACE600} 9.00 \\
						cr & 0.33 & 1.00 & 3.00 & \cellcolor[HTML]{ACE600} 0.20 & \cellcolor[HTML]{ACE600} 0.20 & \cellcolor[HTML]{ACE600} 0.20 & & 0.33 & 1.00 & 3.00 & \cellcolor[HTML]{ACE600} 3.00 & \cellcolor[HTML]{ACE600} 1.00 & \cellcolor[HTML]{ACE600} 5.00 \\
						lt & 0.14 & 0.33 & 1.00 & \cellcolor[HTML]{ACE600} 5.00 & \cellcolor[HTML]{ACE600} 7.00 & 7.00 & & 0.14 & 0.33 & 1.00 & \cellcolor[HTML]{ACE600} 3.00 & \cellcolor[HTML]{ACE600} 3.00 & 7.00 \\
						ob & \cellcolor[HTML]{ACE600} 0.14 & \cellcolor[HTML]{ACE600} 5.00 & \cellcolor[HTML]{ACE600} 0.20 & 1.00 & \cellcolor[HTML]{ACE600} 0.33 & 3.00 & & \cellcolor[HTML]{ACE600} 0.11 & \cellcolor[HTML]{ACE600} 0.33 & \cellcolor[HTML]{ACE600} 0.33 & 1.00 & \cellcolor[HTML]{ACE600} 1.00 & 3.00 \\
						sb & 0.20 & \cellcolor[HTML]{ACE600} 5.00 & \cellcolor[HTML]{ACE600} 0.14 & \cellcolor[HTML]{ACE600} 3.00 & 1.00 & 3.00 & & 0.20 & \cellcolor[HTML]{ACE600} 1.00 & \cellcolor[HTML]{ACE600} 0.33 & \cellcolor[HTML]{ACE600} 1.00 & 1.00 & 3.00 \\
						rp & \cellcolor[HTML]{ACE600} 0.20 & \cellcolor[HTML]{ACE600} 5.00 & 0.14 & 0.33 & 0.33 & 1.00 & & \cellcolor[HTML]{ACE600} 0.11 & \cellcolor[HTML]{ACE600} 0.20 & 0.14 & 0.33 & 0.33 & 1.00
				\end{tabular} \label{tab:pairwiseMatrix}
            \end{center}
	\end{table}

As we can see in this table, the pairwise comparison weights are adjusted primarily according to the weights assigned to the first three criteria. Thus, although there is a change in the pairwise comparison weights, which may be significant, the weights assigned to the first three criteria are preserved. The algorithm will change the weights of the other criteria to force the consistency of the pairwise comparisons matrix. Another significant observation is that the algorithm stops when it reaches the desired consistency rate. So, it is more likely that the algorithm changes the weights related to the fourth criterion and the next ones. Therefore, the ordering of criteria in the matrix must follow the following rules:

\begin{itemize}
\item the three most relevant criteria for the company are among the first;
\item the other criteria are ordered from the last to fourth position, in descending order of relevance.
\end{itemize}

For the data presented in Table 1, there was no concern with the correct ordering of the criteria. Based on the results and previous observations,  we requested the criteria classification according to its relevance to each company. Table 2 presents the modification of results after altering the criteria ordination on the input data.

Speaking about the CR, Tab. 1 clearly shows the inconsistencies before the modification and how the weight modifications made by the algorithm reflect the adjustments
of these inconsistencies. For example, in the case of the plastic bag manufacturing company, the billing criterion is more important than the criticality criterion and much more valuable than the lead time criterion. Also, the criticality criterion is more important than the lead time criterion. These statements are coherent. However, the vector of the weights of this matrix clearly shows that the criticality criterion is less relevant than the lead time criterion, which is inconsistent with the two first statements. The assignment of the other weights related to criteria criticality and lead time causes this inconsistency. We can see from the input data matrix that the obsolescence, replaceability, and repairability criteria are more important than the criticality criterion but less valuable than the lead-time criterion. According to the initial statement (that criticality is more relevant than lead time), this new statement generates the inconsistency of the pairwise comparisons matrix. As we can see in the adjusted matrix, the algorithm quickly corrects these inconsistencies and then the weight vector.




    \section{REFERENCES}
    
    \section{INTRODUÇÃO}
        
        O objetivo destas instruções é servir de guia para a formatação dos trabalhos a serem publicados nos anais. Os anais do XII CONEM serão publicados no formato Adobe$^{\small{TM}}$ PDF.

        Os artigos \textcolor{red}{\underline{\textbf{DEVEM}}} ser formatados de acordo com estas instruções. O presente arquivo pode ser usado como modelo por usuários do \LaTeX. Além disso, pode ser usado como um guia de formatação para usuários de outros processadores de texto.

        Os trabalhos (\textit{draft} e \textit{final paper}) estão limitados a no mínimo \textcolor{red}{\underline{\textbf{6 páginas}}} e no máximo \textcolor{red}{\underline{\textbf{10 páginas}}}, incluindo tabelas e figuras. O arquivo final em formato PDF não deve exceder 2 MB.

        A língua oficial do congresso é o português; entretanto serão aceitos manuscritos em espanhol ou em inglês. Se o trabalho não for escrito em inglês, o autor deverá incluir o título, os nomes dos autores e afiliações, o resumo e as palavras-chave, traduzidos para o inglês, após a lista de referências, no fim do artigo.


    \section{FORMATO DO TEXTO}
        
        O artigo deve ser digitado no \textit{template} ``CONEM2024.tex'' e as principais configurações (ex.: formato do papel, tipo e tamanho da fonte do título/Seção/Subseção, espaços, margens, alinhamentos, indentação, etc.) já estão previamente configuradas no arquivo ``abcm.sty''. Título, nomes dos autores, instituição, endereço, resumo e palavras-chave devem ser recuados 0,1 cm da margem esquerda e marcados por uma linha preta na borda esquerda com largura de 2$\frac{1}{4}$ pontos.
        
        Informações suficientes devem ser fornecidas diretamente no texto, ou por referência a trabalhos publicados disponíveis. Notas de rodapé devem ser evitadas. 
        
        Todos os símbolos e notações devem ser definidos no texto. As grandezas físicas devem ser expressas no Sistema Internacional de Unidades. Os símbolos matemáticos que aparecem no texto devem ser digitados em itálico. As unidades devem ser digitadas no estilo romano (ex.: kg, m, MJ, kW/m$^2$, em vez de $kg$, $m$, $MJ$, $kW/m^2$).
        
        As referências bibliográficas devem ser citadas no texto dando o sobrenome do(s) autor(es) e o ano de publicação, conforme os seguintes exemplos: ``... conforme apresentado \citep{MinwooShamim13}.'' ou ``Recentemente, \citet{MinwooShamim13} demonstraram que …''. No caso de três ou mais autores, deve ser utilizada a expressão ``\textit{et al.}'', como segue: ``... conforme já relatado \citep{Bordalo89}'' e ``Recentemente, \citet{Bordalo89} demonstraram que …''. Duas ou mais referências com os mesmos autores e ano de publicação devem ser distinguidas acrescentando ``a'', ``b'', etc., ao ano de publicação. No caso da necessidade de citar vários trabalhos em uma mesma chamada, o seguinte formato pode ser utilizado: ``...já se encontra consolidado na literatura (\citeauthor{Coimbra78}, \citeyear{Coimbra78}; \citeauthor{Clark86}, \citeyear{Clark86} e \citeauthor{Sparrow80},  \citeyear{Sparrow80})''.  
        
        As referências devem ser listadas no final do manuscrito de acordo com as instruções fornecidas na Seção 4.


        \textbf{\textcolor[rgb]{0.98,0.00,0.00}{\underline{NÃO NUMERAR AS PÁGINAS.}}}

    \subsection{Títulos e Subtítulos das Seções }

        Os títulos das seções são com letras maiúsculas (Exemplo: \textbf{MODELO MATEMÁTICO}), enquanto os subtítulos só têm as primeiras letras maiúsculas de cada palavra (Exemplo: \textbf{Modelo Matemático}). Eles devem ser numerados, usando numerais arábicos separados por pontos, até o máximo de 3 subníveis. Uma linha em branco de espaçamento simples deve ser incluída acima e abaixo de cada título ou subtítulo.

    \subsection{Corpo do Texto}

        Conforme já se encontra definido no arquivo ``abcm.sty'', o corpo do texto é justificado e com espaçamento simples. A primeira linha de cada parágrafo tem recuo de 0,5 cm a partir da margem esquerda.

        As equações matemáticas são alinhadas à esquerda com recuo de 0,5 cm.  Elas são referidas como ``Eq. (\ref{eq:equation1})'' no meio de uma frase, ou ``Equação (\ref{eq:equation1})'' quando usada no início de uma sentença. Os números das equações são numerais arábicos colocados entre parênteses, e alinhados à direita, como mostrado na Eq. (\ref{eq:equation1}). Os símbolos usados nas equações devem ser definidos imediatamente antes ou depois de sua primeira ocorrência no texto.

        O tamanho da fonte usado nas equações deve ser compatível com o utilizado no texto. Todas as grandezas físicas devem ter suas unidades expressas no Sistema Internacional de Unidades, como já mencionado.

        \begin{equation}
        \sigma = E \cdot \epsilon. \label{eq:equation1}
        \end{equation}
    
    	\noindent
    	onde: $\sigma$ é a tensão, $E$ é o módulo de elasticidade longitudinal (ou Módulo de Young) e $\epsilon$ é a deformação longitudinal.

        As figuras e tabelas devem ser colocadas no texto o mais próximo possível do ponto em que foram mencionadas pela primeira vez e devem ser numeradas consecutivamente em algarismos arábicos. As tabelas devem ser centralizadas e referidas por ``Tab. \ref{tab:tabexemplo}'' no meio de uma frase, ou por ``Tabela \ref{tab:tabexemplo}'' quando usada no início de uma sentença. Uma linha em branco, em espaço simples, deve ser introduzida entre a tabela e o texto subsequente.
        
        Anotações e valores numéricos incluídos na tabela devem ter tamanhos compatíveis com a fonte usada no texto do trabalho e todas as unidades devem ser expressas no Sistema Internacional de Unidades. As unidades são incluídas apenas na primeira linha ou primeira coluna de cada tabela, conforme for apropriado. O estilo de borda é livre. As explicações, se houver, devem ser fornecidas ao pé das tabelas e não dentro das mesmas.

        \begin{table}[ht]
            \begin{center}
                \caption{\textbf{Exemplo de tabela.}}
                    \begin{tabular}{c|c|c}
                    \hline
                    Propriedades do compósito       & CFRC-TWILL        & CFRC-4HS         \\
                    \hline
                    Resistência à Flexão  (MPa)     & 209.0 $\pm$ 10.0       & 180.0 $\pm$  15.0    \\
                    \hline
                    Módulo de Flexão  (GPa)         & 57.0 $\pm$ 2.8    & 18.0 $\pm$  1.3  \\
                    \hline
                    \end{tabular} \label{tab:tabexemplo}
            \end{center}
        \end{table}

        As figuras devem ser centralizadas. Elas são referenciadas por ``Fig. \ref{fig:figexemplo}'' no meio de uma frase ou por ``Figura \ref{fig:figexemplo}'' quando usada no início de uma sentença. Sua legenda deve ser centralizada e localizada imediatamente abaixo da figura. As anotações e numerações devem ter tamanhos compatíveis com a fonte usada no texto e todas as unidades devem ser expressas no Sistema Internacional de Unidades. As figuras devem ser colocadas o mais próximo possível de sua primeira citação no texto. Deve ser deixada uma linha em branco, de espaçamento simples, entre a figura e o texto subsequente.
    
        \begin{figure}[h]
            \begin{center}
                \includegraphics[angle=0, scale=.8]{figura.png}
            \end{center}
            \caption{\textbf{Exemplo de figura.}}
            \label{fig:figexemplo}
        \end{figure}

        Figuras coloridas e fotografias de alta qualidade podem ser incluídas no trabalho. Para reduzir o tamanho do arquivo e preservar a resolução gráfica, os arquivos das imagens podem ser convertidos para o formato GIF (para figuras com até 16 cores) ou para o formato JPEG (alta densidade de cores), antes de serem inseridos no trabalho.

    \section{AGRADECIMENTOS}
    
        Se houver, esta seção deve ser colocada antes da lista de referências.


    % REFERÊNCIAS
    \section{REFERÊNCIAS}
    
         %   \citet{key} ==>>                Jones et al. (1990)
         %   \citet*{key} ==>>               Jones, Baker, and Smith (1990)
         %   \citep{key} ==>>                (Jones et al., 1990)
         %   \citep*{key} ==>>               (Jones, Baker, and Smith, 1990)
         %   \citep[chap. 2]{key} ==>>       (Jones et al., 1990, chap. 2)
         %   \citep[e.g.][]{key} ==>>        (e.g. Jones et al., 1990)
         %   \citep[e.g.][p. 32]{key} ==>>   (e.g. Jones et al., p. 32)
         %   \citeauthor{key} ==>>           Jones et al.
         %   \citeauthor*{key} ==>>          Jones, Baker, and Smith
         %   \citeyear{key} ==>>             1990
    
        Referências aceitas incluem: artigos de periódicos, dissertações, teses, artigos publicados em anais de congressos, livros, comunicações privadas e artigos submetidos e aceitos (com fonte identificada) e citações a páginas da internet.

        A lista de referências deve ser uma seção específica denominada Referências, localizada no fim do artigo.

        A primeira linha de cada referência deve ser alinhada à esquerda, sendo que as demais terão recuo já configurado a partir da margem esquerda. Todas as referências incluídas na lista devem aparecer como citações no texto do trabalho.

        As referências devem ser postas em ordem alfabética, usando o último nome do primeiro autor, seguida do ano da publicação. Exemplo da lista de referências é apresentado abaixo.
        
        \bibliographystyle{abcm}
        \bibliography{bibliografia}

    \section{RESPONSABILIDADE AUTORAIS}

        Os trabalhos escritos em português ou espanhol devem incluir (após direitos autorais) título, os nomes dos autores e afiliações, o resumo e as palavras-chave, traduzidos para o inglês e a declaração a seguir, devidamente adaptada para o número de autores.
    
        O(s) autor(es) é(são) o(s) único(s) responsável(is) pelo conteúdo deste trabalho.

 % RESUMO EM INGLES

\noindent{
   \\ 
    \begin{tabular}{||p{\textwidth}}
    \begin{center}
    \vspace{-.6cm}
    \title{AFTER FULL PAPER IN PORTUGUESE OR SPANISH, IT IS NECESSARY THE ABSTRACT IN ENGLISH}
    \end{center}
    \authors{First Author’s Name, e-mail1$^1$} \\
    \authors{Second Author’s Name, e-mail$^2$} \\
    \authors{Third Author’s Name, e-mail$^2$} \\\\
    \institution{$^1$Institution and address for first author} \\
    \institution{$^2$Institution and address for second and third authors} \\
    \\
    \authors{\textcolor[rgb]{0.98,0.00,0.00}{Same format for other authors and institutions, if any. (\textit{This guidance should be removed from the final paper.})}} \\
    \\
    \abstract{\textbf{Abstract:} The purpose of these instructions is to serve as a guide for formatting papers to be published in the Proceedings of the XII CONEM. The abstract must describe the objectives, methodology and main conclusions of the article, containing between 300 and 400 words in a single paragraph.  It should not contain either formulae or bibliographic references. The full paper will be published in the proceedings of the event.}\\
    \\
    \keywords{\textbf{Keywords:} keyword 1, keyword 2, keyword 3 (up to 5 keywords) }\\
    \end{tabular}
}

\end{document}