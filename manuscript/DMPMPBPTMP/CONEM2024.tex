% MODELO CONEM 2022 - Obs. Adequado para o CONEM 2024.
%
% Date: January 12, 2024
%
\documentclass[10pt,fleqn,a4paper,twoside]{article}

\usepackage{algorithm}
\usepackage{algpseudocode}
\usepackage{rotating}
\usepackage{multirow}
\newtheorem{theorem}{Theorem}
\newtheorem{corollary}{Corollary}[theorem]
\newtheorem{lemma}[theorem]{Lemma}
\usepackage{enumitem}
\usepackage{amsfonts} 

\usepackage{array, makecell}

\usepackage{xcolor}
\usepackage{tikz}
\usetikzlibrary{shapes.geometric,shapes.misc,shapes.symbols,arrows.meta,graphs,fit,positioning,shadows}
\usepackage{times}

\usepackage{abcm}
\def\shortauthor{T.B. Fraga, I.R.B. de Aquino and R.C.S. Menêzes}
\def\shorttitle{Multi-period, Multi-product P-batch Processing Time Maximization Problem}

\begin{document}
    
    % CABEÇALHO

\section{Dinamic Multi-period, Multi-product p-Batch Processing Time Maximization Problem - not finished yet}
\label{sec:MBPTMP}

The Multi-product p-Batch Processing Time Maximization Problem (MPpBPTM) problem arises when a set of different products are processed simultaneously in a same production batch. In this problem, it is considered that the quantity produced of each product is directly proportional to the processing time, however, with a different constant of proportionality (production rate) for each product. In addition, there is a maximum quantity allowed for the production of batch products, defined both individually and for the set. The production quantity of each product is mainly defined according to the demand for the product. However, it is still possible to stock the products and / or send them to the outlets. In both cases, there is a stocking / shipping limit for each product and a stocking / shipping limit for the set of products in the batch. Also, there is a time limit available for processing the batch. About the production, the industry manager has as priority to meet demand and he prefer to send the remaining production to outlets rather then stock it in the factory. 

The Dinâmic Multi-period, Multi-product p-Batch Processing Time Maximization (DMPMPpBPTM) problem é uma versão mais complexa do MPpBPTM na qual considera-se um horizonte de planejamento contendo mais do que um único período (dias, semanas, etc). Neste caso é necessário definir o tempo máximo de processamento de cada lote para o período atual, mas considerando um horizonte de planejamento contendo vários períodos. Adicionamente, em um cenário dinâmico, consideramos que já existe um planejamento prévio para produção e entregas. Neste caso o objetivo consiste em definir um encaixe de produção para atender às novas demandas. 

O DMPMPpBPTM é mais complexo do que o MPpBPTM porque, no caso do primeiro, a decisão sobre a produção de determinado produto em determinado período pode influir fortemente nos limites permitidos para produção do mesmo produto e de outros produtos em outros períodos. E o mesmo é válido para os limites de estocagem definidos em cada período.

Para melhor compreensão do problema DMPMPpBPTM e de sua complexidade, considere o problema MPpBPTM com dois produtos e um único período de planejamento (1 dia) apresentado na Tab. 1, e o problema DMPMPpBPTM com dois produtos e um horizonte de planejamento de dois dias, conforme apresentado na Tab. 2, onde: $\textrm{p}_i$ is the production rate of product $i$; $\textrm{UD}_i$ is the demand for the product $i$; $\textrm{UD}_{id}$ is the demand for the product $i$ on day $d$; $\textrm{UO}_i$ is the maximum amount of product $i$ that can be shipped to outlets; $\textrm{UI}_i$ é o limite de estoque para o produto $i$; $\textrm{UI}_{id}$ é o limite de estoque para o produto $i$ no dia $d$; $\textrm{O}$ is the maximum quantity allowed for shipment of all products to outlets; $\textrm{I}$ é o limite de estoque in the factory para all products in the batch, on the first day de planejamento; $\textrm{Z}$ is the timeout for batch processing; $\textrm{Z}_d$ is the timeout for batch processing on day $d$.

\begin{table}[h!]
\begin{center}
\caption{\textbf{Benchmark MPpBPTM 02.}}
\begin{tabular}[c]{c r r c r}
\\
$i$ & 1 & 2 & & total \\
\cline {1-5} \\
$\textrm{p}_i$ & 60 & 40 & \\
$\textrm{UD}_i$ & 1000 & 500 & \\
$\textrm{UO}_i$ & 600 & 600 & $\textrm{O}$ & 1000 \\
$\textrm{UI}_i$ & 3000 & 2000 & $\textrm{I}$ & 3000 \\
\cline {1-5} \\
& & & $\textrm{Z}$ & 100 \\
\end{tabular}
\label{tab:MBPTMP001}
\end{center}
\end{table}

\begin{table}[h!]
\begin{center}
\caption{\textbf{Benchmark DMPMPpBPTM 02.}}
\begin{tabular}[c]{c r r c r}
\\
$i$ & 1 & 2 & & total \\
\cline {1-5} \\
$\textrm{p}_i$ & 60 & 40 & \\
$\textrm{UD}_{i1}$ & 1000 & 500 & \\
$\textrm{UD}_{i2}$ &  200 & 500 & \\
$\textrm{UO}_i$ & 600 & 600 & $\textrm{O}$ & 1000 \\
$\textrm{UI}_{i1}$ & 3000 & 2000 & $\textrm{I}_1$ & 3000 \\
$\textrm{UI}_{i2}$ & 1500 & 2000 & $\textrm{I}_2$ & 1500 \\
\cline {1-5} \\
& & & $\textrm{Z}$ & 100 \\
\end{tabular}
\label{tab:MBPTMP001}
\end{center}
\end{table}

No caso do problema MPpBPTM 02, desejamos encontrar o tempo máximo de planejamento em um único período, T, para o lote contendo os produtos 1 e 2. Neste caso, podemos encontrar T aplicando as equações matemáticas (\ref{eq:01}) e (\ref{eq:02}) propostas por Fraga (2024) para solução deste problema:

\begin{equation}
\label{eq:01}
T^* = \lfloor{\min \{\min_{i} \{(\textrm{UO}_i + \textrm{UI}_i + \textrm{UD}_i) / \textrm{p}_i\},(\textrm{O} + \textrm{I} + \sum_i \textrm{UD}_i) / \sum_i {\textrm{p}_i}\}}\rfloor
\end{equation}

\begin{equation}
\label{eq:02}
T = \min \{T^* , \textrm{Z}\}
\end{equation}

Aplicando estas fórmulas, teremos a solução T = 55. A produção e a distribuição desta produção para atendimento à demanda, envio para outlets e estoque em fábrica são apresentados na Tab. 3.

\begin{table}[h!]
\begin{center}
\caption{\textbf{Results obtained for the MPpBPTM 02.}}
\begin{footnotesize}
\begin{tabular}[c]{c r r r r}
\\
product & production & delivered & send to outlets & stocked in factory \\
\cline {1-5} \\
1 & 3,300 & 1,000 & 400 & 1,900 \\
2 & 2,200 &   500 & 600 & 1,100 \\
\cline {1-5} \\
\end{tabular}
\label{tab:compResultsMBPTMP2}
\end{footnotesize}
\end{center}
\end{table}

Já no caso do problema DMPMPpBPTM 02, apresentado na Tab. 2, é necessário considerar que existe um planejamento prévio para o segundo período e, portanto, variação nos limites de estocagem. Há ainda uma demanda prevista adicional com previsão de entrega para o segundo período.

Por exemplo, no caso do problema apresentado na Tab. 2, antes de considerar qualquer planejamento, temos um limite de estocagem para o produto 2 de 3000 unidade no primeiro dia. Contudo este limite é reduzido para 1500 unidade no segundo dia. Isso pode ser justificado, por exemplo, com uma ordem de produção já lançada 1500 unidados do produto 1 para o segundo período. 

A Tab. 4 apresenta uma atualização dos dados do problema DMPMPpBPTM 02, considerando a solução apresentada para o problema MPpBPTM 02, $T_1 = 55$, e usando a produção do primeiro período para atender a demanda do segundo período. 

\begin{table}[h!]
\begin{center}
\caption{\textbf{Atualização das informação do DMPMPpBPTM 02, após primeiro período com $T_1 = 55$.}}
\begin{tabular}[c]{c r r c r}
\\
$i$ & 1 & 2 & & total \\
\cline {1-5} \\
$\textrm{p}_i$ & 60 & 40 & \\
$\textrm{UD}_{i1}$ & 0 & 0 & \\
$\textrm{UD}_{i2}$ &  0 & 0 & \\
$\textrm{UO}_i$ & 100 & 100 & $\textrm{O}$ & 0 \\
$\textrm{UI}_{i1}$ & 1,100 & 900 & $\textrm{I}_1$ & 0 \\
$\textrm{UI}_{i2}$ & - 200 &  1400 & $\textrm{I}_2$ & -1,000 \\
\cline {1-5} \\
& & & $\textrm{T}_1$ &  55 \\
& & & $\textrm{Z}_2$ &  50 \\
\end{tabular}
\label{tab:MBPTMP001}
\end{center}
\end{table}

Conforme é possível verificar na Tab. 4, considerando $T_1 = 55$, teríamos um excesso de estoque para o segundo período. Este resultado comprova que a determinação do tempo máximo de produção do lote no primeiro período deve levar em conta as informações relativas aos outros períodos. É possível demonstrar que isso é válido para todos os períodos considerados. Contudo tal demonstração não será apresentada neste artigo. 

O problema DMPMPpBPTM pode, portanto, ser definido como base os seguintes aspectos:

\begin{itemize}
\item são considerados $N$ produtos, $i=1,...,N$, que devem ser processados simultaneamente em um único lote;
\item cada produto é produzido de acordo com uma taxa de produção conhecida, $p_i$, a qual pode variar de um produto para outro;
\item é considerado um horizonte de planejamento com $D$ períodos, $d=1,...,D$;
\item o lote será processado no primeiro período;
\item existe uma demanda conhecida para cada produto em cada período, $\textrm{UD}_{id}$;
\item existe um limite conhecido para a quantidade enviada de cada produto para os outlets, $\textrm{UO}_i$; 
\item existe um limite conhecido para a quantidade enviada do conjunto de produtos para os outlets, $\textrm{O}$;
\item as quantidades de cada produto solicitadas diretamente pelos outlets são entendidas como demanda e não influem nos limites anteriormente definidos;
\item existe um limite conhecido para a quantidade de cada produto que pode ser estocada em fábrica, $\textrm{UI}_{id}$, sendo que estes limites podem variar de um período para outro;
\item existe um limite conhecido para a quantidade total do conjunto de produtos que pode ser estocada em fábrica, $\textrm{U}_{d}$, sendo que este limite também pode variar de um período para outro;
\item não é atriuída nenhuma prioridade para quais produtos devem ser estocados em fábrica ou enviados para os outlets;
\item existe um tempo máximo para processamento do lote, $Z$;
\item a produção deve prioritariamente atender à demanda, depois ser enviada para os outlets e, a quantidade restante estocada em fábrica, de acordo com as restrições conhecidas;
\item o problema consiste em definir o máximo tempo viável de processamento para o lote, $T$, e a distribuição da produção resultante, entre atendimento à demanda, envio para outlets e estocagem em fábrica, com base nas restrições acima elencadas;
\end{itemize}

A seção à seguir um modelo metemático para o problema DMPMPpBPTM.

\section{Solving - not finished yet}

No caso de um DMPMPpBPTM, a produção é limitada de acordo com a demanda, o limite da quantidade de cada produto e do cunjunto de produtos que pode ser enviada para os outlets e o limite da quantidade de cada produto e do cunjunto de produtos que pode ser estocada em fábrica.


Observe que, no caso de um DMPMPpBPTM, os dados relativos ao primeiro período estipulam um tempo máximo para processamento do lote, o qual não poderá ser ultrapassado, independente das demandas futuras. Isso porque, apesar de existir a possibilidade de atendimento das demandas futuras com a produção do primeiro período, esta produção deverá ser estocada, respeitando os limites definidos.

Portanto, basicamente, a diferença entre o MPpBPTM e o DMPMPpBPTM está no fato de que poderá haver uma redução no limite considerado para estocagem em fábrica no primeiro período.

Observe que, parte da quantidade estocada em fábrica no primeiro período será utilizada para atendimento à demanda no segundo período, e a quantidade restante, se houver, será mantida em estoque no segundo período. Portanto, podemos definir um novo limite para estocagem em fábrica no primeiro período, $UI_{i1}$, usando a equação:

\begin{equation}
UI_{i1}^* \leq min \{\textrm{UI}_{i1}, \textrm{UI}_{i2} + \textrm{UD}_{i2} \} \quad \forall i
\end{equation}

No caso de considerarmos um terceiro período, a quantidade limite para estoque em fábrica no segundo período seria também afetada pelo terceiro período, assim, a equação (3) pode ser reescrita como:

\begin{equation}
UI_{i1}^* \leq min \{\textrm{UI}_{i1},  UI_{i2}^* +  \textrm{UD}_{i2}\} \quad \forall i
\end{equation}

Estendedo a observação apresentada para um terceiro período, parte da quantidade estocada em fábrica no segundo período será utilizada para atendimento à demanda no terceiro período, e a quantidade restante, se houver, será mantida em estoque no terceiro período. Assim, temos:

\begin{equation}
UI_{i2}^* \leq min \{\textrm{UI}_{i2}, \textrm{UI}_{i3} + \textrm{UD}_{i3} \} \quad \forall i
\end{equation}

Substituindo (5) em (4), temos:

\begin{equation}
UI_{i1}^* \leq min \{\textrm{UI}_{i1},  min \{\textrm{UI}_{i2}, \textrm{UI}_{i3} + \textrm{UD}_{i3} \} +  \textrm{UD}_{i2}\} \quad \forall i
\end{equation}

Usando um pouco de álgebra, temos:

\begin{equation}
UI_{i1}^* \leq min \{\textrm{UI}_{i1}, \textrm{UI}_{i2} + \textrm{UD}_{i2}, \textrm{UI}_{i3} + \textrm{UD}_{i3} +  \textrm{UD}_{i2}\} \quad \forall i
\end{equation}

Assim, generaliznaod para $n$ períodos, teremos a seguinte restrição:

\begin{equation}
UI_{i1}^* \leq min \{\textrm{UI}_{i1}, \min_{d \in \mathbb{Z} , 1 \leq d \leq n} \{\textrm{UI}_{i1}, \textrm{UI}_{id} + \sum_{k=2}^{d} {\textrm{UD}_{ik}}\} \} \quad \forall i
\end{equation}

\subsection{Mathematical Model - not finished yet}
\label{sec:mathModel}

Given that:

$\textrm{UD}_i$ is the demand for the product $i$;

$\textrm{I}$ is the maximum quantity allowed for additional factory storage of all products in the batch;

$\textrm{UI}_i$ is the maximum quantity allowed for stocking the product $i$ in the factory;

$\textrm{O}$ is the maximum quantity allowed for shipment of all products to outlets;

$\textrm{UO}_i$ is the maximum amount of product $i$ that can be shipped to outlets;

$\textrm{p}_i$ is the production rate of product $i$;

$\textrm{Z}$ is the timeout for batch processing;

$P_i$ is the amount of product $i$ produced;

$D_i$ is the amount of product $i$ delivered for the demand;

$O_i$ amount of product $i$ shipped to factory outlets;

$I_i$ is the amount of product $i$ that will be stored at the factory;

$T$ is the batch processing time; \\

We have the problem:

\begin{equation}
\label{MBPTMP01}
max \quad T
\end{equation}

$s.t.$

\begin{equation}
P_i - \textrm{p}_i * T  = 0 \quad \forall i
\end{equation}

\begin{equation}
P_i - D_i - O_i - I_i = 0 \quad \forall i
\end{equation}

\begin{equation}
\label{MBPTMP04}
D_i \leq \textrm{UD}_i \quad \forall i
\end{equation}

\begin{equation}
O_i \leq \textrm{UO}_i \quad \forall i
\end{equation}

\begin{equation}
\sum_i{O_i} \leq \textrm{O}
\end{equation}

\begin{equation}
I_i \leq \textrm{UI}_i \quad \forall i
\end{equation}

\begin{equation}
\sum_i{I_i} \leq \textrm{I}
\end{equation}

\begin{equation}
T \leq \textrm{Z}
\end{equation}

\begin{equation}
\min \{\textrm{UD}_i - D_i, O_i + I_i\} = 0 \quad \forall i
\end{equation}

\begin{equation}
\min\{\textrm{O} - \sum_i\{ O_i\}, \textrm{UO}_i - O_i, I_i\} = 0 \quad \forall i
\end{equation}

\begin{equation}
\label{MBPTMP10}
T, D_i, O_i, I_i \in  \mathbb{Z}^ + \quad \forall i
\end{equation}

where constraints in Eq. (2) relate the quantity produced, $P_i$, to batch processing time $T$. Constraints in Eq. (3) calculate the quantity produced, $P_i$, as a function of the primary variables, $D_i$, $O_i$ and $I_i$. Constraints in Eq. (4), (5), and (7) state that the quantity delivered to demand, the quantity shipped to the outlets, and the factory-stocked quantity of each product must be less than their respective known limits. Constraints deined by Eq. (6) and (8) state that both the sum of product quantities sent to the outlets and the sum of product quantities stocked in the factory must be less than their respective maximum allowed values. The restriction in Eq. (9) establishes that there is a batch processing time limit, $\textrm{Z}$, that must be respected. Equations (10) and (11) define the production distribution priority. And finally, the constraints in Eq. (12) inform the nature of the decision variables.

\subsection{Analytical Solution}
\label{sec:analyticalSol}

The method proposed in this paper for solution of the MPpBPTM problem is composed of two distinct parts, which are: a) calculation of the maximum processing time of the predefined multi-product p-batch; and b) calculation of the part of the production used to meet demand, the part of the production sent to outlets and the part of the production stocked in factory. Next, we explain each of these parts in detail. \\

\noindent \emph{Calculation of maximum p-batch processing time:} \\

It is possible to split the batch processing time into two time slots:

\begin{equation}
T = T' + T''
\end{equation}

\noindent where $T'$ is the maximum processing time whose production will be used exclusively to meet the demand.

Then, being $D_{i}^{'}$ the quantity produced of the product $i$ after $T'$ units of processing time, we have:

\begin{equation}
\label{eq:s02}
D_{i}^{'} - \textrm{p}_i * T'  = 0 \quad \forall i
\end{equation}

\begin{equation}
\label{eq:s03}
D_{i}^{'} \in Z \quad \forall i
\end{equation}

\noindent and, as $D_{i}^{'} \leq D_i$, $\forall i$, we have:

\begin{equation}
\label{eq:s04}
D_{i}^{'} \leq \textrm{UD}_i\quad \forall i
\end{equation}

From the Eq. (\ref{eq:s02}) and the restrictions in Eq. (\ref{eq:s03}) and (\ref{eq:s04}), we have:

\begin{equation}
\label{eq:s05}
T' \leq \lfloor{\textrm{UD}_i / \textrm{p}_i}\rfloor \quad \forall i
\end{equation}

And, as we wish maximum value to $T'$, from Eq. (\ref{eq:s05}) we have:

\begin{equation}
\label{eq:s06}
T' = \min_i\{\lfloor{\textrm{UD}_i / \textrm{p}_i}\rfloor\}
\end{equation}

Observe that there may be a leftover corresponding to the demand not met by the production in the first part of the processing time, $T'$. This leftover, $S_i$, will be defined according to the equations in Eq. (\ref{eq:s07}):

\begin{equation}
\label{eq:s07}
S_i = \textrm{UD}_i - D_{i}^{'} \quad \forall i
\end{equation}

It is also possible to consider that outlets are extensions of the factory stock. Therefore, it is possible to consider the factory stock of product $i$ and the outlets stock $i$ as single stock for the product $i$, $E_i$, so that:

\begin{equation}
\label{eq:s08}
E_i = O_i + I_i
\end{equation}

\begin{equation}
\label{eq:s09}
E_i \leq \textrm{UO}_i + \textrm{UI}_i
\end{equation}

\begin{equation}
\label{eq:s10}
\sum_i {E_i} \leq \textrm{O} + \textrm{I}
\end{equation}

\begin{equation}
\label{eq:s11}
E_i \in Z \quad \forall i
\end{equation}

As the time $T'$ was used only for production that will meet the demand, $T''$  will be the remaining processing time, \emph{i.e.} the maximum processing time used for production that will meet the leftover of demand, sent to the outlets and stored in factory. So we have:

\begin{equation}
\label{eq:s12}
E_i + S_i - \textrm{p}_i * T''  = 0 \quad \forall i
\end{equation}

\noindent Thus, after replacing Eq. (\ref{eq:s12}) in Eq. (\ref{eq:s09}) and applying some algebra, we have:

\begin{equation}
\label{eq:s13}
T''  \leq (\textrm{UO}_i + \textrm{UI}_i + S_i) / \textrm{p}_i \quad \forall i
\end{equation}

\noindent And, after replacing Eq. (\ref{eq:s12}) in Eq. (\ref{eq:s10}) and applying some algebra, we have:

\begin{equation}
\label{eq:s14}
 T'' \leq (\textrm{O} + \textrm{I} + \sum_i S_i) / \sum_i \textrm{p}_i
\end{equation}

Finally, as $T''$  is an integer value and we want to maximize this value, then:

\begin{equation}
\label{eq:s15}
 T'' = \lfloor\min\{\min_i\{ (\textrm{UO}_i + \textrm{UI}_i + S_i) / \textrm{p}_i\}, (\textrm{O} + \textrm{I} + \sum_i S_i) / \sum_i \textrm{p}_i \}\rfloor
\end{equation}

So, the analytical method for finding maximum $T$ can be decomposed into 4 steps:

step 1: find $T'$, where:
\begin{equation}
T' = \lfloor{\min_{i} \{\textrm{UD}_i / \textrm{p}_i\}}\rfloor
\end{equation}

step 2: calculate $S_i$ for all products.
\begin{equation}
\label{eq:unmet}
S_i = \textrm{UD}_i - \textrm{p}_i * T' \quad \forall i
\end{equation}

step 3: find $T''$, where:
\begin{equation}
T'' = \lfloor{\min \{\min_{i} \{(\textrm{UO}_i + \textrm{UI}_i + S_i) / \textrm{p}_i\},(\textrm{O} + \textrm{I} + \sum_i {S_i}) / \sum_i {\textrm{p}_i}\}}\rfloor
\end{equation}

step 4: calculate $T$, where: 
\begin{equation}
T^* = T' + T''
\end{equation}
\begin{equation}
T = \min \{T^* , \textrm{Z}\}
\end{equation}

Note that, as we reduce the value of $T'$, the consequent reduction of $D_{i}^{'}$ will be directly attributed to the leftover of demand $S_i$ and, therefore, there will be no change in the optimal solution if it is considered to $T'$, a value between 0 and the value found in step 1. Thus it is possible to use a simplification of the method presented, making $T' = 0$. In this case $S_i = \textrm{UD}_i, \forall i$. The simplified version of the method is presented below:

step 1: find $T^*$, where:
\begin{equation}
T^* = \lfloor{\min \{\min_{i} \{(\textrm{UO}_i + \textrm{UI}_i + \textrm{UD}_i) / \textrm{p}_i\},(\textrm{O} + \textrm{I} + \sum_i \textrm{UD}_i) / \sum_i {\textrm{p}_i}\}}\rfloor
\end{equation}

step 2: calculate $T$, where: 

\begin{equation}
T = \min \{T^* , \textrm{Z}\}
\end{equation}

This simplification of the method demonstrates that the value calculated to $T$ does not depend on the way the production of the product $i$, $P_i$, is distributed between service to demand, $D_i$, outlets, $O_i$, and  factory's stock, $I_i$.

\noindent \emph{Full solution:} 

We can determine now the values of $D_i$, $O_i$ and $I_i$, $\forall i$, simply attending the priorities for distribution. The following three algorithms can be used to find a complete solution: The Algorithm (\ref{alg:part01}) is used for finding an optimimal value for $T$. Then an initial production distribution can be done by the Algorithm (\ref{alg:part02}). This distribution is made without considering the restrictions defined for the set of products. $SO$ and $SI$ represent, respectively, the total amount that can be sent to the outlets and the total amount that can be stocked in factory after the initial distribution. If this values are negative, this indicates that the proposed initial distribution is not a feasible distribution. So, after applying Algorithm (\ref{alg:part02}), if $SO \geq 0$ and $SI \geq 0$, solution found is an optimal solution. However, if $SO < 0$ or $SI <0$, solution is not yet feasible. But as the solution found to $T$ is feasible to the problem defined in Eq. (\ref{MBPTMP01}) to (\ref{MBPTMP10}), we have the following statements: \\

$if \quad SO < 0 \quad then \quad SI > 0 \quad and  \quad |SO| < SI$ \\

$if \quad SI < 0 \quad then \quad SO > 0 \quad and  \quad |SI| < SO$ \\

\begin{algorithm}[h!]
\caption{Solving MPpBPTM problem | Part 01 - find an optimal $T$ to the problem defined in (\ref{MBPTMP01}) to (\ref{MBPTMP10}).}\label{alg:part01}
\begin{algorithmic}
	\Require $\textrm{UD}_i, \textrm{UO}_i, \textrm{UI}_i, \textrm{p}_i, \forall i, \textrm{O}, \textrm{I}, \textrm{Z}$
	\State $T^* \gets \lfloor{\min \{\min_{i} \{(\textrm{UO}_i + \textrm{UI}_i + \textrm{UD}_i) / \textrm{p}_i\},(\textrm{O} + \textrm{I} + \sum_i \textrm{UD}_i) / \sum_i {\textrm{p}_i}\}}\rfloor$
	\State $T \gets \min \{T^* , \textrm{Z}\}$
\State \textbf{Return:} $T$.
\end{algorithmic}
\end{algorithm}

\begin{algorithm}
\caption{Solving MPpBPTM problem | Part 02 - calculate $D_i$, $O_i$ and $I_i$, $\forall i$, ignoring the restrictions for the set of products of the batch.}\label{alg:part02}
\begin{algorithmic}
	\Require $\textrm{UD}_i, \textrm{UO}_i, \textrm{UI}_i, \textrm{p}_i, \forall i, \textrm{O}, \textrm{I}, \textrm{Z}, T$.
	\For{all i}
		\State $E_i \gets T * \textrm{p}_i$
		\State $D_i \gets \min \{\textrm{UD}_i, E_i\}$
		\State $E_i \gets E_i - D_i$
		\State $O_i \gets \min \{\textrm{UO}_i, E_i\}$
		\State $E_i = E_i - O_i$
		\State $\textrm{UO}_i = \textrm{UO}_i - O_i$
		\State $I_i = E_i$
		\State $\textrm{UI}_i = \textrm{UI}_i - I_i$
	\EndFor
	\State $SO = \textrm{O} - \sum_i{O_i}$
	\State $SI = \textrm{I} - \sum_i{I_i}$
	\State \textbf{Return:} $D_i, O_i, I_i, E_i, \textrm{UO}_i, \textrm{UI}_i, \forall i, SO, SI$.
\end{algorithmic}
\end{algorithm}

Therefore it is possible to meet the restrictions by applying the Agorithm (\ref{alg:part03}). This algorithm redistributes the production of factory's stock to the outlets or vice versa, respectively, due to having a $SI$ or $SO$ negative.

\begin{algorithm}[h!]
\caption{Solving MPpBPTM problem | Part 03: redistribute production to comply with limitation restrictions for the batch set.}\label{alg:part03}
\begin{algorithmic}
\Require $O_i, I_i, \textrm{UO}_i, \textrm{UI}_i, \forall i, SO, SI$.
\If{$SO < 0$}
\For{all $i$}
	\State $O_i \gets O_i - \min\{ O_i, UI_i, |SO|\}$
	\State $I_i \gets I_i + \min\{ O_i, UI_i, |SO|\}$
	\State $SO \gets SO + \min\{ O_i, UI_i, |SO|\}$
	\If{$SO = 0$}
		break looping for;
	\EndIf
\EndFor
\EndIf

\If{$SI < 0$}
\For{all $i$}
	\State $I_i \gets I_i - \min\{ I_i, UO_i, |SI|\}$
	\State $O_i \gets O_i + \min\{ I_i, UO_i, |SI|\}$
	\State $SI \gets SI + \min\{ I_i, UO_i, |SI|\}$
	\If{$SI = 0$}
		break looping for;
	\EndIf
\EndFor
\EndIf
\State \textbf{Return:} $O_i, I_i, \forall i$.
\end{algorithmic}
\end{algorithm}

\section{RESULTS AND DISCUSSION}
\label{sec:results}

To test the developed model and analytical solution method, we created a solver in LINGO named MPpBPTM.lng and also a solver in C++ named COPSolver: library for solving the multi-product p-batch processing time maximization problem, both available at codeocean and GitHub. A brief explanation of the software and links for downloads can be found in the paper of \cite{Fraga2023}. The tests were performed on a notebook with an Intel i7 processor. We tested the solvers developed for the benchmarks presented on Tab. \ref{tab:MBPTMP001}, \ref{tab:MBPTMP002} and \ref{tab:MBPTMP003}, and for randon benchmarks generated by the following functions:

\begin{equation}
\textrm{p}_i = \textrm{rand}()\%30 + 10
\end{equation}

\begin{equation}
\textrm{UD}_i = \textrm{rand}()\%3000 + 800;
\end{equation}

\begin{equation}
 \textrm{seed1} = \textrm{rand}()\%3000 + 500;
\end{equation}

\begin{equation}
\textrm{seed2} = \textrm{rand}()\%5000 + 1000;
\end{equation}

\begin{equation}
\textrm{O} = \textrm{N}/2*\textrm{seed1};
\end{equation}

\begin{equation}
\textrm{UO}_i = \textrm{rand}()\%(\textrm{seed1}-500) + 500;
\end{equation}

\begin{equation}
\textrm{I} = \textrm{N}/2*\textrm{seed2};
\end{equation}

\begin{equation}
\textrm{UI}_i = \textrm{rand}()\%(\textrm{seed2}-1000) + 1000;
\end{equation}

\begin{equation}
\textrm{Z} = 100;
\end{equation}

Randomly generated benchmarks were named RMPpBPTM $\textrm{N}$, being $\textrm{N}$ the number of products. Seeking to enable the reproduction of the results, in the computational construction of the benchmarks we used the function srand((unsigned) source), where source is a defined value. To build the results presented in this work, we used source=0. Files containing benchmarks used for the tests performed and all results can be found at tbfraga.github.io/COPSolver/benchmarks.

\begin{table}[h!]
\begin{center}
\caption{\textbf{Benchmark MPpBPTM 2.}}
\begin{tabular}[c]{c r r r}
\\
$i$ & 1 & 2 & total \\
\cline {1-4} \\
$\textrm{p}_i$ & 60 & 40 & \\
$\textrm{UD}_i$ & 1000 & 500 & \\
$\textrm{UO}_i$ & 600 & 600 & 1000 \\
$\textrm{UI}_i$ & 3000 & 2000 & 3000 \\
\cline {1-4} \\
& & $\textrm{Z}$ & 100 \\
\end{tabular}
\label{tab:MBPTMP001}
\end{center}
\end{table}

\begin{table}[h!]
\begin{center}
\caption{\textbf{Benchmark MPpBPTM 3.}}
\begin{tabular}[c]{c r r r r}
\\
$i$ & 1 & 2 & 3 & total \\
\cline {1-5} \\
$\textrm{p}_i$ & 60 & 40 & 50 \\
$\textrm{UD}_i$ & 1000 & 500 & 800 \\
$\textrm{UO}_i$ & 600 & 600 & 600 & 1500 \\
$\textrm{UI}_i$ & 3000 & 2000 & 1000 & 3500 \\
\cline {1-5} \\
& & & $\textrm{Z}$ & 100 \\
\end{tabular}
\label{tab:MBPTMP002}
\end{center}
\end{table}

\begin{table}[h!]
\begin{center}
\caption{\textbf{Benchmark MPpBPTM 10.}}
\begin{small}
\begin{tabular}[c]{c r r r r r r r r r r r }
\\
$i$ & 1 & 2 & 3 & 4 & 5 & 6 & 7 & 8 & 9 & 10 & total \\
\cline {1-12} \\
$\textrm{p}_i$ & 60 & 40 & 50 & 40 & 30 & 50 & 60 & 10 & 20 & 40\\
$\textrm{UD}_i$ & 1000 & 500 & 800 & 500 & 400 & 500 & 2000 & 300 & 500 & 1000 \\
$\textrm{UO}_i$ & 600 & 600 & 600 & 1500 & 300 & 200 & 500 & 800 & 0 & 200 & 3000 \\
$\textrm{UI}_i$ & 3000 & 2000 & 1000 & 800 & 3000 & 1000 & 400 & 300 & 200 & 0 & 5000 \\
\cline {1-12} \\
& & & & & & & & & & $\textrm{Z}$ & 100 \\
\end{tabular}
\label{tab:MBPTMP003}
\end{small}
\end{center}
\end{table}

Table \ref{tab:results} presents the results obtained by applying the analytical method (COPSolver) and the LINGO solver for the solution of the previously presented benchmarks. Tables (\ref{tab:compResultsMBPTMP2}) and (\ref{tab:compResultsMBPTMP3}) shows the production distribution for MPpBPTM 3 and 10 found by both COPSolver and LINGO solver. It is important to point out that the LINGO solver was developed with the purpose of validating the results found by the proposed analytical method. It is possible to verify that both solvers are capable of finding optimal solutions for the proposed benchmarks very quickly, even for very large-scale problems. However, as the analytical method is a polynomial time complexity method, it is much more efficient. In addition, the modular structure of the library developed for COPSolver allows the easy application of the proposed solution method when developing new heuristics for solving MPpBS problems, making this tool a significant technological contribution to the production management of some industries as well as for researchers in the area. Since we have considered here a static problem, in a future work we will study the complexities of solving the problem dynamically and try to propose new solution methods if needed. 

\begin{table}[h!]
\begin{center}
\caption{\textbf{Results obtained with the LINGO solver and the analytical method.}}
\begin{footnotesize}
\begin{tabular}[c]{l r r r r}
\\
problem & LINGO solver & time (s) & analytical method & time (s) \\
\cline {1-5} \\
MPpBPTM 2 & 55 & 0.03 & 55 & $<$ 0.01 \\
MPpBPTM 3 & 48 & 0.03 & 48 & $<$ 0.01 \\
MPpBPTM 10 & 30 & 0.05 & 30 & $<$ 0.01 \\
RMPpBPTM 20 & 100 & 0.06 & 100 & $<$ 0.01 \\
RMPpBPTM 50 & 98 & 0.08 & 98 & $<$ 0.01 \\
RMPpBPTM 100 & 98 & 0.11 & 98 & $<$ 0.01 \\
RMPpBPTM 1,000 & 78 & 1.20 & 78 & $<$ 0.01 \\
RMPpBPTM 2,000 & 70 & 3.25 & 70 & $<$ 0.01 \\
RMPpBPTM 5,000 & 70 & 15.12 & 70 & $<$ 0.02 \\
RMPpBPTM 10,000 & 70 & 55.19 & 70 & $<$ 0.04 \\
\cline {1-5} \\
\end{tabular}
\label{tab:results}
\end{footnotesize}
\end{center}
\end{table}

\begin{table}[h!]
\begin{center}
\caption{\textbf{Results obtained with the LINGO solver and COPSolver for the MPpBPTM 3.}}
\begin{footnotesize}
\begin{tabular}[c]{l r r r r}
\\
product & production & delivered & send to outlets & stocked in factory \\
\cline {1-5} \\
P1 & 2,880 & 1,000 & 300 & 1,580 \\
P2 & 1,920 & 500 & 600 & 820 \\
P3 & 2,400 & 800 & 600 & 1,000 \\
\cline {1-5} \\
\end{tabular}
\label{tab:compResultsMBPTMP2}
\end{footnotesize}
\end{center}
\end{table}

\begin{table}[h!]
\begin{center}
\caption{\textbf{Results obtained with the LINGO solver and COPSolver for the MPpBPTM 10.}}
\begin{footnotesize}
\begin{tabular}[c]{l r r r r}
\\
product & production & delivered & send to outlets & stocked in factory \\
\cline {1-5} \\
P1 & 1,800 & 1,000 & 400 & 400 \\
P2 & 1,200 & 500 & 600 & 100 \\
P3 & 1,500 & 800 & 600 & 100 \\
P4 & 1,200 & 500 & 700 & 0 \\
P5 & 900 & 400 & 300 & 200 \\
P6 & 1,500 & 500 & 200 & 800 \\
P7 & 1,800 & 1800 & 0 & 0 \\
P8 & 300 & 300 & 0 & 0 \\
P9 & 600 & 500 & 0 & 100 \\
P10 & 1,200 & 1000 & 200 & 0 \\
\cline {1-5} \\
\end{tabular}
\label{tab:compResultsMBPTMP3}
\end{footnotesize}
\end{center}
\end{table}

\section{CONCLUSIONS AND SUGGESTIONS FOR FUTURE WORKS}
\label{sec:conclusions}

In this paper we presented the multi-product p-batch processing time maximization problem, as well as a mathematical model and an exact analytical solution method for this problem. The mathematical model and analytical method were tested, respectively, by a solver developed with the LINGO software, from LINDO Systems, and with a solver developed in C++ language. Optimum results were found for all proposed benchmarks, very quickly, even in the case of very large-scale problems, which demonstrates the efficiency of the proposed method. As in this paper we considered a static problem, in a future work we will study the complexities that arise when considering the same problem in a dinamic scenario. We will also verify if there is the possibility of extending the proposed analytical method to solve a class of integer programming problems with similar characteristics to the studied problem.

\section{CREDIT AUTHORSHIP CONTRIBUTION STATEMENT} 
\label{sec:contributions}

T.B. Fraga: Conceptualization, Methodology, Software, Validation, Formal analysis, Investigation, Data Curation, Writing – original draft, Writing – review \& editing, Supervision, Project administration. Í.R.B. Aquino: Investigation. R.C.S. Menêzes: Investigation.

\section{ACKNOWLEDGMENTS}
\label{sec:acknowledgments}

We are enormously grateful to Coordenação de Aperfeiçoamento de Pessoal de Nível Superior (CAPES) and to Conselho Nacional de Desenvolvimento Científico e Tecnológico (CNPq) for the financial support provided to our projects. We also thank LINDO systems team for the LINGO software license, without which this work would not have been possible and the to the owner of the company in the plastics sector, who allowed us to learn about his company's production process. Finally, we would like to thank Pró-reitoria de Extensão e Cultura da UFPE (PROExC) and the Research Director of Propesqi (Pró-reitoria de Pesquisa e Inovação da UFPE) for their support and recognition of our work, and dear Professors Antônio José da Silva Neto and João Flávio Vieira de Vasconcellos from IPRJ/UERJ, who contributed significantly to the formation of essential skills for the development of our projects. We also thank my co-worker Marcos Luiz Henrique, for having helped by evaluating the mathematical model and solution method proposed in this paper.

    % REFERÊNCIAS
   \section{REFERENCES}
    
         %   \citet{key} ==>>                Jones et al. (1990)
         %   \citet*{key} ==>>               Jones, Baker, and Smith (1990)
         %   \citep{key} ==>>                (Jones et al., 1990)
         %   \citep*{key} ==>>               (Jones, Baker, and Smith, 1990)
         %   \citep[chap. 2]{key} ==>>       (Jones et al., 1990, chap. 2)
         %   \citep[e.g.][]{key} ==>>        (e.g. Jones et al., 1990)
         %   \citep[e.g.][p. 32]{key} ==>>   (e.g. Jones et al., p. 32)
         %   \citeauthor{key} ==>>           Jones et al.
         %   \citeauthor*{key} ==>>          Jones, Baker, and Smith
         %   \citeyear{key} ==>>             1990
    
        
        \bibliographystyle{abcm}
        \bibliography{bibliografia}
        
\section{COPYRIGHT LIABILITY}
    
        The authors are uniquely responsible for the content of this work.

\end{document}