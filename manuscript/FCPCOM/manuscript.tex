%% 
%% Copyright 2007-2020 Elsevier Ltd
%% 
%% This file is part of the 'Elsarticle Bundle'.
%% ---------------------------------------------
%% 
%% It may be distributed under the conditions of the LaTeX Project Public
%% License, either version 1.2 of this license or (at your option) any
%% later version.  The latest version of this license is in
%%    http://www.latex-project.org/lppl.txt
%% and version 1.2 or later is part of all distributions of LaTeX
%% version 1999/12/01 or later.
%% 
%% The list of all files belonging to the 'Elsarticle Bundle' is
%% given in the file `manifest.txt'.
%% 
%% Template article for Elsevier's document class `elsarticle'
%% with harvard style bibliographic references

\documentclass[authoryear,manuscript,12pt]{elsarticle}
\usepackage{setspace}
\doublespacing
\usepackage{algorithm}
\usepackage{algpseudocode}
%\usepackage{algorithm2e}
\usepackage{rotating}
\usepackage{multirow}
\newtheorem{theorem}{Theorem}
\newtheorem{corollary}{Corollary}[theorem]
\newtheorem{lemma}[theorem]{Lemma}
\usepackage{enumitem}

\usepackage{array, makecell}

\usepackage{xcolor}
\usepackage{tikz}
\usetikzlibrary{shapes.geometric,shapes.misc,shapes.symbols,arrows.meta,graphs,fit,positioning,shadows}



%% The `ecrc' package must be called to make the CRC functionality available
%\usepackage{ecrc}

%% Use the option review to obtain double line spacing
%% \documentclass[authoryear,preprint,review,12pt]{elsarticle}

%% Use the options 1p,twocolumn; 3p; 3p,twocolumn; 5p; or 5p,twocolumn
%% for a journal layout:
%% \documentclass[final,1p,times,authoryear]{elsarticle}
%% \documentclass[final,1p,times,twocolumn,authoryear]{elsarticle}
%% \documentclass[final,3p,times,authoryear]{elsarticle}
%% \documentclass[final,3p,times,twocolumn,authoryear]{elsarticle}
%% \documentclass[final,5p,times,authoryear]{elsarticle}
%% \documentclass[final,5p,times,twocolumn,authoryear]{elsarticle}

%% For including figures, graphicx.sty has been loaded in
%% elsarticle.cls. If you prefer to use the old commands
%% please give \usepackage{epsfig}

%% The amssymb package provides various useful mathematical symbols
\usepackage{amssymb}
%% The amsthm package provides extended theorem environments
%% \usepackage{amsthm}

%% The lineno packages adds line numbers. Start line numbering with
%% \begin{linenumbers}, end it with \end{linenumbers}. Or switch it on
%% for the whole article with \linenumbers.
%% \usepackage{lineno}

\journal{Computers \& Industrial Engineering}

\begin{document}

\begin{frontmatter}
%% Title, authors and addresses

%% use the tnoteref command within \title for footnotes;
%% use the tnotetext command for theassociated footnote;
%% use the fnref command within \author or \affiliation for footnotes;
%% use the fntext command for theassociated footnote;
%% use the corref command within \author for corresponding author footnotes;
%% use the cortext command for theassociated footnote;
%% use the ead command for the email address,
%% and the form \ead[url] for the home page:
%% \title{Title\tnoteref{label1}}
%% \tnotetext[label1]{}
%% \author{Name\corref{cor1}\fnref{label2}}
%% \ead{email address}
%% \ead[url]{home page}
%% \fntext[label2]{}
%% \cortext[cor1]{}
%% \affiliation{organization={},
%%            addressline={}, 
%%            city={},
%%            postcode={}, 
%%            state={},
%%            country={}}
%% \fntext[label3]{}

\title{How demand pattern identification and multicriteria ABC classification can guide management decisions}

%% use optional labels to link authors explicitly to addresses:
%% \author[label1,label2]{}
%% \affiliation[label1]{organization={},
%%             addressline={},
%%             city={},
%%             postcode={},
%%             state={},
%%             country={}}
%%
%% \affiliation[label2]{organization={},
%%             addressline={},
%%             city={},
%%             postcode={},
%%             state={},
%%             country={}}

%\author{Tatiana Balbi Fraga, Ítalo Ruan Barbosa de Aquino and Regilda da Costa e Silva Menêzes}

\author{Tatiana Balbi Fraga\corref{cor1}\fnref{label1}}
\ead{tatiana.balbi@ufpe.br}
\cortext[cor1]{corresponding author}

\author{Beatriz Marinho Cavalcanti\fnref{label1}}
\ead{<beatriz.marinhocavalcanti@ufpe.br}

\author{Alexia Maria Duque Silva\fnref{label1}}
\ead{alexia.duque@ufpe.br}

\author{Erika Leticia Rodrigues Silva\fnref{label1}}
\ead{erika.leticias@ufpe.br}

%\affiliation[label1]{
%			 organization={Centro Acadêmico do Agreste - Universidade Federal de Pernambuco},
%             addressline={Avenida Marielle Franco, Bairro Nova Caruaru},
%             city={Caruaru},
%             postcode={55014-900},
%             state={PE},
%             country={Brasil} }

\affiliation[label1]{
		   organization={Agreste Academic Center - Federal University of Pernambuco},%Department and Organization
           addressline={Avenida Marielle Franco, Nova Caruaru}, 
           city={Caruaru},
           postcode={55014-900}, 
           state={PE},
           country={Brazil}}

\begin{abstract} 
Multicriteria classification is usualy very important to the decision-making in manufacturing management process. For such classification, the attribution of weights to the criteria strongly influences the coherence of the results found. Saaty's Analytic Hierarchy Process (AHP) is an important method for assigning weights to multiple criteria.  AHP's logic is not complicated at all but, since matrices of pairwise comparisons of criteria  are usually generated manually and based only on some employee know-how, there is a huge complexity on generating a consistent pairwise matrix. Especially when many criteria are used. This paper presents a constructive algorithm that can be used to adjust inconsistent matrices, forcing such matrices to have a better consistency rate. We tested this algorithm by applying the AHP method, for multicriteria ABC classification, to companies in two sectors. As a result we observed that the algorithm can adjust the pairwise matrices in just a few seconds, avoiding the manual work that would be done in weeks, therefore showing that it is an important resource for applying the AHP method. We also present in this paper an analysis of the importance of the attribution of the weights to the criteria and show how the multicriteria and demand pattern classifications may influence the decision on the choice of the appropriate forecasting method.
\end{abstract}
%%Graphical abstract
%\begin{graphicalabstract}
%\end{graphicalabstract}

%%Research highlights
\begin{highlights}
\item literature review about multicriteria classification;
\item literature review about demand pattern classification;
\item generative method to force consistency of pairwise comparisons matrix;
\item importance of the correct balance of product criteria weights assigning;
\item importance of ABC multicriteria and demand pattern classifications for forecasting;
\item brief description of COPSolver: library for solving the multicriteria classification problem;
\item brief description of COPSolver: library for solving the demand pattern classification problem.
\end{highlights}

\begin{keyword}
demand pattern identitification \sep multicriteria ABC classification \sep analytic hierarchy process \sep pairwise matrix consistency \sep COPSolver
\end{keyword}
\end{frontmatter}

%% \linenumbers

%% main text
\section{Introduction}
\label{sec:intro}


\section{Algorithm for forcing pairwise matrix consistency}


\section{Tests and results}
\label{sec:results}

To test the two developed COPSolver libraries (COPSolver: library for solving the multi-criteria classification problem and COPSolver: library for solving the demand pattern classification problem), we used data from three companies in three different sectors (plastic packaging manufacturers, furniture retailers and car mechanics). In the case of the company in the car mechanics sector, only the data relating to truck repairs was used; in the case of the other two companies, all the data obtained from all the products sold over the last 5 years was used. The files containing the formatted data used for all the tests can be found at tbfraga.github.io/COPSolver/benchmarks.

\section{Conclusions and suggestions for future works}
\label{sec:conclusions}

In this paper we presented ...

\section{CRediT authorship contribution statement} 
\label{sec:contributions}

T.B. Fraga: Conceptualization, Project administration, Supervision, Software, Methodology, Validation, Formal analysis, Writing – original draft, Writing – review \& editing. 

\section{Acknowledgments}
\label{sec:acknowledgments}

%% The Appendices part is started with the command \appendix;
%% appendix sections are then done as normal sections
%% \appendix

%% \section{}
%% \label{}

%% If you have bibdatabase file and want bibtex to generate the
%% bibitems, please use
%%
%%  \bibliographystyle{elsarticle-harv} 
%%  \bibliography{<your bibdatabase>}

%% else use the following coding to input the bibitems directly in the
%% TeX file.

\begin{thebibliography}{1}

%% \bibitem[Author(year)]{label}
%% Text of bibliographic item

\bibitem[\protect\citeauthoryear{Flores and Whybark}{1987}]{FloresWhybark1987}
Flores, B. E., Whybark, D. C. (1987). Implementing Multiple Criteria ABC Analysis. {\it Journal of Operations Management}, Vol. 7 (1,2), pp. 79--85.

\bibitem[\protect\citeauthoryear{Fraga}{2023}]{Fraga2023}
Fraga, T.B. (2023). COPSolver: open source software for solving combinatorial optimization and other decision problems - library for solving the multicriteria classification problem, in press.

\end{thebibliography}
\end{document}

\endinput
%%
%% End of file `elsarticle-template-harv.tex'.
