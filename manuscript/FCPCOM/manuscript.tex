%% 
%% Copyright 2007-2020 Elsevier Ltd
%% 
%% This file is part of the 'Elsarticle Bundle'.
%% ---------------------------------------------
%% 
%% It may be distributed under the conditions of the LaTeX Project Public
%% License, either version 1.2 of this license or (at your option) any
%% later version.  The latest version of this license is in
%%    http://www.latex-project.org/lppl.txt
%% and version 1.2 or later is part of all distributions of LaTeX
%% version 1999/12/01 or later.
%% 
%% The list of all files belonging to the 'Elsarticle Bundle' is
%% given in the file `manifest.txt'.
%% 
%% Template article for Elsevier's document class `elsarticle'
%% with harvard style bibliographic references

\documentclass[authoryear,manuscript,12pt]{elsarticle}
\usepackage{setspace}
\doublespacing
\usepackage{algorithm}
\usepackage{algpseudocode}
%\usepackage{algorithm2e}
\usepackage{rotating}
\usepackage{multirow}
\newtheorem{theorem}{Theorem}
\newtheorem{corollary}{Corollary}[theorem]
\newtheorem{lemma}[theorem]{Lemma}
\usepackage{enumitem}

\usepackage{array, makecell}

\usepackage[table]{xcolor}
\usepackage{tikz}
\usetikzlibrary{shapes.geometric,shapes.misc,shapes.symbols,arrows.meta,graphs,fit,positioning,shadows}



%% The `ecrc' package must be called to make the CRC functionality available
%\usepackage{ecrc}

%% Use the option review to obtain double line spacing
%% \documentclass[authoryear,preprint,review,12pt]{elsarticle}

%% Use the options 1p,twocolumn; 3p; 3p,twocolumn; 5p; or 5p,twocolumn
%% for a journal layout:
%% \documentclass[final,1p,times,authoryear]{elsarticle}
%% \documentclass[final,1p,times,twocolumn,authoryear]{elsarticle}
%% \documentclass[final,3p,times,authoryear]{elsarticle}
%% \documentclass[final,3p,times,twocolumn,authoryear]{elsarticle}
%% \documentclass[final,5p,times,authoryear]{elsarticle}
%% \documentclass[final,5p,times,twocolumn,authoryear]{elsarticle}

%% For including figures, graphicx.sty has been loaded in
%% elsarticle.cls. If you prefer to use the old commands
%% please give \usepackage{epsfig}

%% The amssymb package provides various useful mathematical symbols
\usepackage{amssymb}
%% The amsthm package provides extended theorem environments
%% \usepackage{amsthm}

%% The lineno packages adds line numbers. Start line numbering with
%% \begin{linenumbers}, end it with \end{linenumbers}. Or switch it on
%% for the whole article with \linenumbers.
%% \usepackage{lineno}

\journal{Computers \& Industrial Engineering}

\begin{document}

\begin{frontmatter}
%% Title, authors and addresses

%% use the tnoteref command within \title for footnotes;
%% use the tnotetext command for theassociated footnote;
%% use the fnref command within \author or \affiliation for footnotes;
%% use the fntext command for theassociated footnote;
%% use the corref command within \author for corresponding author footnotes;
%% use the cortext command for theassociated footnote;
%% use the ead command for the email address,
%% and the form \ead[url] for the home page:
%% \title{Title\tnoteref{label1}}
%% \tnotetext[label1]{}
%% \author{Name\corref{cor1}\fnref{label2}}
%% \ead{email address}
%% \ead[url]{home page}
%% \fntext[label2]{}
%% \cortext[cor1]{}
%% \affiliation{organization={},
%%            addressline={}, 
%%            city={},
%%            postcode={}, 
%%            state={},
%%            country={}}
%% \fntext[label3]{}

\title{How demand pattern identification and multicriteria ABC classification can guide management decisions}

%% use optional labels to link authors explicitly to addresses:
%% \author[label1,label2]{}
%% \affiliation[label1]{organization={},
%%             addressline={},
%%             city={},
%%             postcode={},
%%             state={},
%%             country={}}
%%
%% \affiliation[label2]{organization={},
%%             addressline={},
%%             city={},
%%             postcode={},
%%             state={},
%%             country={}}

%\author{Tatiana Balbi Fraga, Ítalo Ruan Barbosa de Aquino and Regilda da Costa e Silva Menêzes}

\author{Tatiana Balbi Fraga\corref{cor1}\fnref{label1}}
\ead{tatiana.balbi@ufpe.br}
\cortext[cor1]{corresponding author}

\author{Beatriz Marinho Cavalcanti\fnref{label1}}
\ead{<beatriz.marinhocavalcanti@ufpe.br}

\author{Alexia Maria Duque Silva\fnref{label1}}
\ead{alexia.duque@ufpe.br}

\author{Erika Leticia Rodrigues Silva\fnref{label1}}
\ead{erika.leticias@ufpe.br}

%\affiliation[label1]{
%			 organization={Centro Acadêmico do Agreste - Universidade Federal de Pernambuco},
%             addressline={Avenida Marielle Franco, Bairro Nova Caruaru},
%             city={Caruaru},
%             postcode={55014-900},
%             state={PE},
%             country={Brasil} }

\affiliation[label1]{
		   organization={Agreste Academic Center - Federal University of Pernambuco},%Department and Organization
           addressline={Avenida Marielle Franco, Nova Caruaru}, 
           city={Caruaru},
           postcode={55014-900}, 
           state={PE},
           country={Brazil}}

\begin{abstract} 

Multicriteria classification and demand pattern identification are two very important techniques that can be used for decision-making in manufacturing management process. Although there are many studies that address different multicriteria techniques, few publications associate the use of multicriteria classification with the identification of demand patterns. They are complementary techniques, which however are rarely addressed together. In this article we present a case study in which we designed product portfolios applying the ABC / Saaty's Analytic Hierarchy Process (AHP) multicriteria classification, in conjunction with the Willians demand pattern identification method, to support the inventory control process and sales strategy in three small and medium-sized companies, from three different sectors. Based on the results found, we present an analysis of the impact of the identified patterns on production and inventory planning, and present inventory management decisions and sales strategies that can be taken based on the portfolio built to mitigate the impact of product demand behavior on production efficiency and lead times. In carrying out this work, the matrices of pairwise comparisons of criteria were generated manually and based only on employees know-how, so we identified a strong difficulty in the construction of consistent matrices, so we also present a constructive algorithm that can be used to adjust inconsistent matrices when applying AHP, forcing such matrices to have a better consistency rate.  As a result we observed that this algorithm could adjust the pairwise matrices in just a few seconds, avoiding the manual work that would be done in weeks, showing, therefore, that it is an important resource when applying the AHP method.

\end{abstract}
%%Graphical abstract
%\begin{graphicalabstract}
%\end{graphicalabstract}

%%Research highlights
\begin{highlights}
\item literature review about multicriteria classification;
\item literature review about demand pattern classification;
\item generative method to force consistency of pairwise comparisons matrix;
\item importance of the correct balance of product criteria weights assigning;
\item importance of ABC multicriteria and demand pattern classifications for forecasting;
\item brief description of COPSolver: library for solving the multicriteria classification problem;
\item brief description of COPSolver: library for solving the demand pattern classification problem.
\end{highlights}

\begin{keyword}
demand pattern identitification \sep multicriteria ABC classification \sep analytic hierarchy process \sep pairwise matrix consistency \sep COPSolver
\end{keyword}
\end{frontmatter}

%% \linenumbers

%% main text
\section{Introduction}
\label{sec:intro}

Decades of experience in applying the philosophy of lean production eventually popularized the famous saying: "stock is a necessary evil" \citep{Younkin2021}. On the one hand the stock of raw material, and finished and semi-finished goods serves to reduce the delivery time for the customer and ensure that possible failures in the process do not generate greater influence on production efficiency as well as delays in deliveries, on the other hand the stock usually represents a high cost, allocation of spaces that could be used for other purposes, possibilities of loss and depreciation, in addition to hiding process failures \citep{Shingo1996}. Briefly, it is safe to say that stock is unavoidable and must be kept up with in suitable amount, and the well-organized management of raw materials and supplies, semi-finished goods and finished goods is nominated Inventory Control \citep{Nirmala2022}.

As this topic is of utmost importance for companies, the scientific literature presents a large number of works proposing different techniques for Inventory Control (\emph{e.g.} incluir referências sobre trabalhos propondo novas técnicas de controle de inventário) and it is a unanimous conclusion that a multicriteria classification is an essential technique when inventory control involves a high quantity of different goods (incluir referências que dão suporte à frase), which is usually what happens in practice. 

Falar sobre técnicas de controle de estoque (classificação ABC (multicriterio) e identificação de padrão de demanda).

Referências sobre classificação ABC multicriteŕio. (revisões da literatura). Dar enfoque ao ABC multicritério com AHP. \cite{FloresEtAl1992} suggest the use of Analytic Hierarchy Process (AHP) to integrate the use of several criteria and rank inventory items. Dezde então, diversos trabalhos tem utilizado esta técnica. Falar sobre o artigo que compara AHP e algoritmos genéticos.

Referências sobre identificação de padrão de demanda. (revisão da literatura). Dar enfoque ao método de Willians.

Apresentar as contribuições científicas deste artigo.

Also, in multicriteria ABC classification, the attribution of weights to the criteria strongly influences the coherence of the results found. Saaty's Analytic Hierarchy Process (AHP) is an important method for assigning weights to multiple criteria.  AHP's logic is not complicated at all but, since matrices of pairwise comparisons of criteria  are usually generated manually and based only on some employee know-how, there is a huge complexity on generating a consistent pairwise matrix. Especially when many criteria are used. This paper presents a constructive algorithm that can be used to adjust inconsistent matrices, forcing such matrices to have a better consistency rate. We tested this algorithm by applying the AHP method, for multicriteria ABC classification, to companies in two sectors. As a result we observed that the algorithm can adjust the pairwise matrices in just a few seconds, avoiding the manual work that would be done in weeks, therefore showing that it is an important resource for applying the AHP method. We also present in this paper an analysis of the importance of the attribution of the weights to the criteria and show how the multicriteria and demand pattern classifications may influence the decision on the choice of the appropriate forecasting method.



\section{Algorithm for forcing pairwise matrix consistency}

\section{COPSolver}

\section{Data collected}
\label{sec:data}

To test the two developed COPSolver libraries (COPSolver: library for solving the multicriteria classification problem and COPSolver: library for solving the demand pattern classification problem), we used data from three companies from three different sectors (plastic packaging manufacturing, furniture trades and car mechanics). In the case of the company from the car mechanics sector, only the data relating to truck repairs was used; in the case of the other two companies, all the data obtained from all the products sold over 5 years (last 60 months prior to the data collection date) was used. One year's data (last 12 months prior to the data collection date) was used for the ABC multicriteria classification. The files containing the formatted data used for all the tests and the results can be found at tbfraga.github.io/COPSolver/benchmarks. 

\section{Tests and results}
\label{sec:results}

\subsection{Pairwise comparisons matrix and consistency rate}

Table \ref{tab:pairwiseMatrix} presents the modifications made by the software 'COPSolver: library for solving the multicriteria classification problem' in the pairwise comparisons matrix and the consequent change in the consistency rate (CR) for the data colected from the three companies selected for this study.

\begin{table}[h!]
\begin{center}
\begin{small}
\begin{tabular}[l]{p{0.4cm} p{0.5cm} p{0.5cm} p{0.5cm} p{0.5cm} p{0.5cm} p{0.5cm} p{0.5cm} | p{0.5cm} p{0.5cm} p{0.5cm} p{0.5cm} p{0.5cm} p{0.5cm} p{0.5cm}}
\multicolumn{14}{l}{car mechanics company} \\
\cline {1-15} \\
 & \multicolumn{7}{c | }{CR = 0.216}  & \multicolumn{7}{c}{CR = 0.085} \\
   & sb & lt & rp & cr & ob & np & cm & sb & lt & rp & cr & ob & co & cm \\
sb & 1.00 & 3.00 & 3.00 & \cellcolor[HTML]{ACE600} 5.00 & 7.00 & 9.00 & 9.00 & 1.00 & 3.00 & 3.00 & \cellcolor[HTML]{ACE600} 3.00 & 7.00 & 9.00 & 9.00 \\
lt & 0.33 & 1.00 & 3.00 & \cellcolor[HTML]{ACE600} 5.00 & \cellcolor[HTML]{ACE600} 7.00 & 9.00 & 9.00 & 0.33 & 1.00 & 3.00 & \cellcolor[HTML]{ACE600} 3.00 & \cellcolor[HTML]{ACE600} 5.00 & 9.00 & 9.00 \\
rp & 0.33 & 0.33 & 1.00 & \cellcolor[HTML]{ACE600} 3.00 & \cellcolor[HTML]{ACE600} 9.00 & \cellcolor[HTML]{ACE600} 9.00 & 9.00 & 0.33 & 0.33 & 1.00 & \cellcolor[HTML]{ACE600} 1.00 & \cellcolor[HTML]{ACE600} 3.00 & \cellcolor[HTML]{ACE600} 3.00 & 9.00 \\
cr & \cellcolor[HTML]{ACE600} 0.20 & \cellcolor[HTML]{ACE600} 0.20 & \cellcolor[HTML]{ACE600} 0.33 & 1.00 & \cellcolor[HTML]{ACE600} 5.00 & \cellcolor[HTML]{ACE600} 9.00 & 9.00 & \cellcolor[HTML]{ACE600} 0.33 & \cellcolor[HTML]{ACE600} 0.33 & \cellcolor[HTML]{ACE600} 1.00 & 1.00 & \cellcolor[HTML]{ACE600} 3.00 & \cellcolor[HTML]{ACE600} 3.00 & 9.00 \\
ob & 0.14 & \cellcolor[HTML]{ACE600} 0.14 & \cellcolor[HTML]{ACE600} 0.11 & \cellcolor[HTML]{ACE600} 0.20 & 1.00 & \cellcolor[HTML]{ACE600} 5.00 & 9.00 & 0.14 & \cellcolor[HTML]{ACE600} 0.20 & \cellcolor[HTML]{ACE600} 0.33 & \cellcolor[HTML]{ACE600} 0.33 & 1.00 & \cellcolor[HTML]{ACE600} 1.00 & 9.00 \\
np & 0.11 & 0.11 & \cellcolor[HTML]{ACE600} 0.11 & \cellcolor[HTML]{ACE600} 0.11 & \cellcolor[HTML]{ACE600} 0.20 & 1.00 & 9.00 & 0.11 & 0.11 & \cellcolor[HTML]{ACE600} 0.33 & \cellcolor[HTML]{ACE600} 0.33 & \cellcolor[HTML]{ACE600} 1.00 & 1.00 & 9.00 \\
cm & 0.11 & 0.11 & 0.11 & 0.11 & 0.11 & 0.11 & 1.00 & 0.11 & 0.11 & 0.11 & 0.11 & 0.11 & 0.11 & 1.00 \\
\cline {1-15} \\
\multicolumn{14}{l}{furniture trades company} \\
\cline {1-15} \\
   & \multicolumn{7}{c | }{CR = 0.151}  & \multicolumn{7}{c}{CR = 0.096} \\
   & lt & sb & rp & cr & cm & np & ob & lt & sb & rp & cr & cm & np & ob \\
lt & 1.00 & 0.20 & 0.14 & \cellcolor[HTML]{ACE600} 0.20 & 7.00 & \cellcolor[HTML]{ACE600} 0.11 & 5.00 & 1.00 & 0.20 & 0.14 & \cellcolor[HTML]{ACE600} 0.33 & 7.00 & \cellcolor[HTML]{ACE600} 0.20 & 5.00 \\
sb & 5.00 & 1.00 & 0.33 & \cellcolor[HTML]{ACE600} 0.33 & 5.00 &\cellcolor[HTML]{ACE600}  0.14 & 7.00 & 5.00 & 1.00 & 0.33 & \cellcolor[HTML]{ACE600} 1.00 & 5.00 & \cellcolor[HTML]{ACE600} 0.33 & 7.00 \\
rp & 7.00 & 3.00 & 1.00 & \cellcolor[HTML]{ACE600} 0.33 & 5.00 & \cellcolor[HTML]{ACE600} 0.14 & 3.00 & 7.00 & 3.00 & 1.00 & \cellcolor[HTML]{ACE600} 1.00 & 7.00 & \cellcolor[HTML]{ACE600} 1.00 & 3.00 \\
cr & \cellcolor[HTML]{ACE600} 5.00 & \cellcolor[HTML]{ACE600} 3.00 & \cellcolor[HTML]{ACE600} 3.00 & 1.00 & \cellcolor[HTML]{ACE600} 9.00 & \cellcolor[HTML]{ACE600} 0.33 & 9.00 & \cellcolor[HTML]{ACE600} 3.00 & \cellcolor[HTML]{ACE600} 1.00 & \cellcolor[HTML]{ACE600} 1.00 & 1.00 & \cellcolor[HTML]{ACE600} 7.00 & \cellcolor[HTML]{ACE600} 1.00 & 9.00 \\
cm & 0.14 & 0.20 & \cellcolor[HTML]{ACE600} 0.20 & \cellcolor[HTML]{ACE600} 0.11 & 1.00 & 0.11 & 1.00 & 0.14 & 0.20 & \cellcolor[HTML]{ACE600} 0.14 & \cellcolor[HTML]{ACE600} 0.14 & 1.00 & 0.11 & 1.00 \\
np & \cellcolor[HTML]{ACE600} 9.00 & \cellcolor[HTML]{ACE600} 7.00 & \cellcolor[HTML]{ACE600} 7.00 & \cellcolor[HTML]{ACE600} 3.00 & 9.00 & 1.00 & 9.00 & \cellcolor[HTML]{ACE600} 5.00 & \cellcolor[HTML]{ACE600} 3.00 & \cellcolor[HTML]{ACE600} 1.00 & \cellcolor[HTML]{ACE600} 1.00 & 9.00 & 1.00 & 9.00 \\
ob & 0.20 & 0.14 & 0.33 & 0.11 & 1.00 & 0.11 & 1.00 & 0.20 & 0.14 & 0.33 & 0.11 & 1.00 & 0.11 & 1.00 \\
\cline {1-15} \\
\multicolumn{14}{l}{plastic packaging manufacturing company} \\
\cline {1-15} \\
   & \multicolumn{7}{c | }{CR = 0.625}  & \multicolumn{7}{c}{CR = 0.085} \\
   & np & cr & lt & ob & sb & rp & & np & cr & lt & ob & sb & rp & \\
np & 1.00 & 3.00 & 7.00 & \cellcolor[HTML]{ACE600} 7.00 & 5.00 & \cellcolor[HTML]{ACE600} 5.00 & & 1.00 & 3.00 & 7.00 & \cellcolor[HTML]{ACE600} 9.00 & 5.00 & \cellcolor[HTML]{ACE600} 9.00 \\
cr & 0.33 & 1.00 & 3.00 & \cellcolor[HTML]{ACE600} 0.20 & \cellcolor[HTML]{ACE600} 0.20 & \cellcolor[HTML]{ACE600} 0.20 & & 0.33 & 1.00 & 3.00 & \cellcolor[HTML]{ACE600} 3.00 & \cellcolor[HTML]{ACE600} 1.00 & \cellcolor[HTML]{ACE600} 5.00 \\
lt & 0.14 & 0.33 & 1.00 & \cellcolor[HTML]{ACE600} 5.00 & \cellcolor[HTML]{ACE600} 7.00 & 7.00 & & 0.14 & 0.33 & 1.00 & \cellcolor[HTML]{ACE600} 3.00 & \cellcolor[HTML]{ACE600} 3.00 & 7.00 \\
ob & \cellcolor[HTML]{ACE600} 0.14 & \cellcolor[HTML]{ACE600} 5.00 & \cellcolor[HTML]{ACE600} 0.20 & 1.00 & \cellcolor[HTML]{ACE600} 0.33 & 3.00 & & \cellcolor[HTML]{ACE600} 0.11 & \cellcolor[HTML]{ACE600} 0.33 & \cellcolor[HTML]{ACE600} 0.33 & 1.00 & \cellcolor[HTML]{ACE600} 1.00 & 3.00 \\
sb & 0.20 & \cellcolor[HTML]{ACE600} 5.00 & \cellcolor[HTML]{ACE600} 0.14 & \cellcolor[HTML]{ACE600} 3.00 & 1.00 & 3.00 & & 0.20 & \cellcolor[HTML]{ACE600} 1.00 & \cellcolor[HTML]{ACE600} 0.33 & \cellcolor[HTML]{ACE600} 1.00 & 1.00 & 3.00 \\
rp & \cellcolor[HTML]{ACE600} 0.20 & \cellcolor[HTML]{ACE600} 5.00 & 0.14 & 0.33 & 0.33 & 1.00 & & \cellcolor[HTML]{ACE600} 0.11 & \cellcolor[HTML]{ACE600} 0.20 & 0.14 & 0.33 & 0.33 & 1.00
\end{tabular}
\caption{Pairwise comparisons matrix and consistency rate changes for three companies (results found by COPSolver) - legend: sb = substitutability; lt = lead-time; rp = repairability; cr = criticality; ob = obsolescence; np = net-profit; cm = commonality;}
\label{tab:pairwiseMatrix}
\end{small}
\end{center}
\end{table} 

As we can see in this table, the pairwise comparison weights are adjusted primarily according to the  weights assigned to the first three criteria. Thus, although there is a change in the pairwise comparison weights, which may be significant, the weights assigned to the first three criteria will be preserved. The change will be made in the weights of the other criteria in order to force the consistency of the pairwise comparisons matrix. Another important point to note is that the algorithm stops when the desired consistency rate is reached. So, it is more likely that there will be significant changes in the fourth criterion and the next ones. Therefore, it is important that the criteria are ordered in such a way that the three most relevant criteria for the company are among the first. Then the next most relevant criteria can be ordered from last to fourth position.

\subsection{Analysis of the ABC multicriteria classification}

\subsection{Analysis of the demand pattern classification}

\begin{table}[h!]
\begin{center}
\begin{small}
\begin{tabular}[c]{l r r | l r r r r}
\multicolumn{8}{l}{car mechanics company} \\
\cline {1-8} \\
\multicolumn{3}{l |}{multc. ABC clssf.} & \multicolumn{5}{c}{demand pattern classification}\\
\multicolumn{3}{c |}{cutoff}  & \multicolumn{5}{c}{75 sales between October 2020 and July 2023} \\
\multicolumn{3}{c |}{0.35 \%; 0.70 \%} & & \multicolumn{1}{c}{total} & \multicolumn{1}{c}{A} & \multicolumn{1}{c}{B} & \multicolumn{1}{c}{C}\\
A &  10 & 23.81 \% & smooth & 18.75 \% &  20.00 \% & 16.67 \% & 0 \%\\
B &  15 & 35.71 \% & slow-moving & 6.25 \% & 10.00 \% & 0.00 \% & 0 \%\\
C &  17 & 40.48 \% & sporadic & 75.00 \% & 70.00 \% & 83.33 \% & 0 \% \\
\cline {1-8} \\
\multicolumn{8}{l}{furniture trades company} \\
\cline {1-8} \\
\multicolumn{3}{l |}{multc. ABC clssf.} & \multicolumn{5}{c}{demand pattern classification}\\
\multicolumn{3}{c |}{cutoff:} &\\
\multicolumn{3}{c |}{0.35 \%; 0.70 \%}& & \multicolumn{1}{c}{total} & \multicolumn{1}{c}{A} & \multicolumn{1}{c}{B} & \multicolumn{1}{c}{C}\\
A &  26 & 10.24 \% & smooth & 31.58 \% &  73.08 \% & 30.00 \% & 24.82 \%\\
B &  80 & 31.50 \% & slow-moving & 18.62 \% & 19.23 \% & 33.75 \% & 9.93 \%\\
C & 148 & 58.27 \% & sporadic & 49.80 \% & 7.69 \% & 36.25 \% & 65.25 \% \\
\cline {1-8} \\
 \multicolumn{8}{l}{plastic packaging manufacturing company}\\
\cline {1-8} \\
\multicolumn{3}{l |}{multc. ABC clssf.} & \multicolumn{5}{c}{demand pattern classification}\\
\multicolumn{3}{c |}{cutoff:} & \multicolumn{5}{c}{5,967 sales between November 2019 and February 2023}\\
\multicolumn{3}{c |}{0.35 \%; 0.70 \%} & & \multicolumn{1}{c}{total} & \multicolumn{1}{c}{A} & \multicolumn{1}{c}{B} & \multicolumn{1}{c}{C}\\
A &   128 &  5.49 \% & smooth & 15.31 \% & 22.66 \% & 23.97 \% & 10.42 \%\\
B &   726 & 31.12 \% & slow-moving & 5.53 \% & 1.56 \% & 8.68 \% & 4.33 \%\\
C & 1,479 & 63.39 \% & sporadic & 79.16 \% & 75.78 \% & 67.36 \% & 85.25 \% \\
\cline {1-8} \\

\end{tabular}
\caption{ABC multicriteria and demand pattern classifications for three companies (results found by COPSolver)}
\label{tab:results}
\end{small}
\end{center}
\end{table} 



\section{Conclusions and suggestions for future works}
\label{sec:conclusions}

In this paper we presented ...

\section{CRediT authorship contribution statement} 
\label{sec:contributions}

T.B. Fraga: Conceptualization, Project administration, Supervision, Software, Methodology, Validation, Formal analysis, Writing – original draft, Writing – review \& editing. 

\section{Acknowledgments}
\label{sec:acknowledgments}

%% The Appendices part is started with the command \appendix;
%% appendix sections are then done as normal sections
%% \appendix

%% \section{}
%% \label{}

%% If you have bibdatabase file and want bibtex to generate the
%% bibitems, please use
%%
%%  \bibliographystyle{elsarticle-harv} 
%%  \bibliography{<your bibdatabase>}

%% else use the following coding to input the bibitems directly in the
%% TeX file.

\begin{thebibliography}{1}

%% \bibitem[Author(year)]{label}
%% Text of bibliographic item

\bibitem[\protect\citeauthoryear{Flores et al.}{1992}]{FloresEtAl1992}
Flores, B. E., Olson, D. L., Dorai V. K. (1992). Management of multicriteria inventory classification. {\it Mathematical and Computer Modelling}, Vol. 16 (12), pp. 71--82.

\bibitem[\protect\citeauthoryear{Flores and Whybark}{1987}]{FloresWhybark1987}
Flores, B. E., Whybark, D. C. (1987). Implementing Multiple Criteria ABC Analysis. {\it Journal of Operations Management}, Vol. 7 (1,2), pp. 79--85.

\bibitem[\protect\citeauthoryear{Fraga}{2023}]{Fraga2023}
Fraga, T.B. (2023). COPSolver: open source software for solving combinatorial optimization and other decision problems - library for solving the multicriteria classification problem, in press.

\bibitem[\protect\citeauthoryear{Younkin}{2021}]{Younkin2021}
Younkin, K. (2021). The Effect of Lean Inventory and Policies during COVID-19. {\it linkedin}, https://www.linkedin.com/pulse/effect-lean-inventory-policies-during-covid-19-kyle-younkin (last accessed on 09/01/2024).

\bibitem[\protect\citeauthoryear{Nirmala}{2022}]{Nirmala2022}
Nirmala, D. A. R., Kannan, V., Thanalakshmi, M., Gnanaraj, S. J. P., Appadurai, M. (2022). Inventory management and control system using ABC and VED analysis. {\it Materials Today: Proceedings}, Vol. 60 (2), pp. 922--925.

\bibitem[\protect\citeauthoryear{Shingo}{1996}]{Shingo1996}
Shingo, S. (1996). {\it O Sistema Toyota de Produção do ponto de vista da Engenharia de Produção}, tradução Eduardo Schaan, Ed. 2, Porto Alegre: Bookman.

\end{thebibliography}
\end{document}

\endinput
%%
%% End of file `elsarticle-template-harv.tex'.
